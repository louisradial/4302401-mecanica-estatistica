\begin{lemma}{Integral de Bose-Einstein}{integral_bose_einstein}
    Para todo \(0 < z \leq 1\) e \(s > 0\), vale
    \begin{equation*}
        \int_0^\infty \dli{x} \frac{x^{s-1}}{\frac{e^x}{z} - 1} = \begin{cases}
            \Li{s}(z)\Gamma(s),&\text{se }s \neq 1\\
            -\ln(1 - z),&\text{se }s = 1
        \end{cases},
    \end{equation*}
    onde \(\Li{s}(z) = \sum_{n\in \mathbb{N}}\frac{z^n}{n^s}\) é a função polilogarítmica e \(\Gamma(s) = \int_0^\infty\dli{x} x^{s-1} e^{-x}\) é a função gama.
\end{lemma}
\begin{proof}
    Seja \(z \in (0,1]\) e \(s > 0\), com \(s \neq 1\). Então
    \begin{align*}
        \int_0^\infty \dli{x} \frac{x^{s-1}}{\frac{e^x}{z} - 1} &= \int_0^\infty\dli{x} x^{s-1} \sum_{n = 1}^\infty \left(\frac{e^{x}}{z}\right)^{-n} &
        \int_0^\infty \dli{x} \frac{1}{\frac{e^x}{z} - 1} &= \int_0^\infty\dli{x} \sum_{n = 1}^\infty \left(\frac{e^{x}}{z}\right)^{-n} \\
        &= \sum_{n = 1}^\infty z^{n} \int_0^\infty \dli{x}{x}^{s - 1}e^{-nx} &
        &= \sum_{n = 1}^\infty z^{n} \int_0^\infty \dli{x}e^{-nx} \\
        &= \sum_{n = 1}^\infty \frac{z^n}{n^s} \int_0^\infty \dli{x} x^{s - 1} e^{-x}&
        &= \sum_{n = 1}^\infty \frac{z^n}{n} \int_0^\infty \dli{x} e^{-x}\\
        &= \sum_{n=1}^\infty \frac{z^n}{n^s} \Gamma(s)&
        &= \sum_{n=1}^\infty \frac{z^n}{n}\\
        &= \Li{s}(z)\zeta(s)&
        &= -\ln(1 - z),
    \end{align*}
    como desejado.
\end{proof}
\begin{exercício}{Gás de Fótons e lei de Stefan-Boltzmann}{exercício1}
    O espectro de energia de um gás de fótons num espaço \(d\)-dimensional de volume \(L^d\) é dado pela expressão \(\mathcal{H} = \sum_{\vetor{k}, j} \hbar \omega_{\vetor{k}, j}(n_{\vetor{k}, j} + \frac12)\), onde \(j\) indica a polarização, \(\vetor{k}\) é o vetor de onda e \(\omega_{\vetor{k},j} = c\norm{\vetor{k}}\) é a relação de dispersão, onde \(c\) é a velocidade da luz.
    \begin{enumerate}[label=(\alph*)]
        \item Calcule a função de partição canônica associada a este hamiltoniano.
        \item Mostre que a energia interna é dada pela expressão \(U = \sigma V T^n\) e determine os valores das constantes \(n\) e \(\sigma\).
    \end{enumerate}
\end{exercício}
\begin{proof}[Resolução]
    Para cada par \(j, \vetor{k}\), temos
    \begin{equation*}
        \Xi_{\vetor{k},j} = \sum_{n = 0}^\infty e^{-\beta \hbar c \norm{\vetor{k}} \left(n + \frac12\right)} = e^{-\frac12\beta \hbar c \norm{\vetor{k}}} \sum_{n=0}^\infty e^{- \beta \hbar c \norm{\vetor{k}} n} = \frac{e^{-\frac12 \beta \hbar c \norm{\vetor{k}}}}{1 - e^{-\beta\hbar c \norm{\vetor{k}}}}
    \end{equation*}
    portanto a grande função de partição é dada por
    \begin{align*}
        \Xi = \prod_{\vetor{k},j} \Xi_{\vetor{k},j} = \prod_{\vetor{k},j} \frac{e^{-\frac12 \beta \hbar c \norm{\vetor{k}}}}{1 - e^{-\beta \hbar c \norm{\vetor{k}}}}
        &\implies \ln \Xi = \sum_{\vetor{k}, j} \left[-\frac12 \beta \hbar c \norm{\vetor{k}} - \ln\left(1 - e^{- \beta \hbar c \norm{\vetor{k}}}\right)\right]\\
        &\implies \ln \Xi = -\frac12\sum_{\vetor{k}, j} \beta \hbar c \norm{\vetor{k}} - \sum_{\vetor{k},j} \ln\left(1 - e^{- \beta \hbar c \norm{\vetor{k}}}\right).
    \end{align*}
    O primeiro termo se refere a energia do vácuo, portanto doravante consideraremos o segundo termo, isto é, a função de partição dada por
    \begin{equation*}
        \ln \tilde{\Xi} = - \sum_{\vetor{k}, j} \ln\left(1 - e^{-\beta \hbar c \norm{\vetor{k}}}\right)
    \end{equation*}
    e, como não há acoplamento com a polarização, temos
    \begin{equation*}
        \ln \tilde\Xi = -\ell \sum_{\vetor{k}} \ln\left(1 - e^{- \beta \hbar c \norm{\vetor{k}}}\right),
    \end{equation*}
    pois \todo[há \(\ell\) modos de polarização.]

    A energia interna é dada por
    \begin{equation*}
        U = - \diffp{\ln \tilde\Xi}{\beta} = \ell \sum_{\vetor{k}} \frac{\hbar c \norm{\vetor{k}} e^{-\beta \hbar c \norm{\vetor{k}}}}{1 - e^{-\beta \hbar c \norm{\vetor{k}}}} = \frac{\ell V}{(2\pi)^d} \int_{\mathbb{R}^3}\dln{d}{k} \frac{\hbar c \norm{\vetor{k}}}{e^{\beta \hbar c \norm{\vetor{k}}}-1}
    \end{equation*}
    onde escrevemos \(V = L^d\) para o volume. Como feito na \href{https://github.com/louisradial/4302401-mecanica-estatistica/releases/tag/lista4}{Lista IV}\footnote{Ver Lema 2.}, temos
    \begin{equation*}
        U = \frac{2\pi^{\frac{d}{2}}\ell V}{(2\pi)^d \Gamma\left(\frac{d}{2}\right)} \int_0^\infty \dli{k} \frac{\hbar c k^d}{e^{\beta \hbar c k}-1} = \frac{2\pi^{\frac{d}{2}} \ell V}{\beta^{d + 1}(h c)^d \Gamma\left(\frac{d}{2}\right)} \int_0^\infty \dli{\xi} \frac{\xi^d}{e^\xi - 1} = \sigma V T^{n},
    \end{equation*}
    onde \(n = d + 1\) e
    \begin{equation*}
        \sigma = \frac{2\pi^{\frac{d}{2}} k_B^{d + 1}\ell}{(hc)^d \Gamma\left(\frac{d}{2}\right)}\int_0^\infty \dli{\xi} \frac{\xi^d}{e^{\xi} - 1} = \frac{2\pi^{\frac{d}{2}}k_B \ell \Gamma(d + 1) \zeta(d + 1)}{\Gamma\left(\frac{d}{2}\right)} \left(\frac{k_B}{hc}\right)^d,
    \end{equation*}
    pelo \cref{lem:integral_bose_einstein}, com \(\Li{s}(1) = \zeta(s)\) para todo \(s > 1\).
\end{proof}
