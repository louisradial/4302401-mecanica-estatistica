\begin{exercício}{Gás de Fótons e lei de Stefan-Boltzmann}{exercício1}
    O espectro de energia de um gás de fótons num espaço \(d\)-dimensional de volume \(L^d\) é dado pela expressão \(\mathcal{H} = \sum_{\vetor{k}, j} \hbar \omega_{\vetor{k}, j}(n_{\vetor{k}, j} + \frac12)\), onde \(j \in \set{1,2}\) indica a polarização, \(\vetor{k}\) é o vetor de onda e \(\omega_{\vetor{k},j} = c\norm{\vetor{k}}\) é a relação de dispersão, onde \(c\) é a velocidade da luz.
    \begin{enumerate}[label=(\alph*)]
        \item Calcule a função de partição canônica associada a este hamiltoniano.
        \item Mostre que a energia interna é dada pela expressão \(U = \sigma V T^n\) e determine os valores das constantes \(n\) e \(\sigma\).
    \end{enumerate}
\end{exercício}
\begin{proof}[Resolução]

\end{proof}
