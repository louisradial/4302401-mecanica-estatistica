\begin{exercício}{Gás de Fótons e lei de Stefan-Boltzmann}{exercício1}
    O espectro de energia de um gás de fótons num espaço \(d\)-dimensional de volume \(L^d\) é dado pela expressão \(\mathcal{H} = \sum_{\vetor{k}, j} \hbar \omega_{\vetor{k}, j}(n_{\vetor{k}, j} + \frac12)\), onde \(j \in \set{1,2}\) indica a polarização, \(\vetor{k}\) é o vetor de onda e \(\omega_{\vetor{k},j} = c\norm{\vetor{k}}\) é a relação de dispersão, onde \(c\) é a velocidade da luz.
    \begin{enumerate}[label=(\alph*)]
        \item Calcule a função de partição canônica associada a este hamiltoniano.
        \item Mostre que a energia interna é dada pela expressão \(U = \sigma V T^n\) e determine os valores das constantes \(n\) e \(\sigma\).
    \end{enumerate}
\end{exercício}
\begin{proof}[Resolução]
    Para cada par \(j, \vetor{k}\), temos
    \begin{equation*}
        \Xi_{\vetor{k},j} = \sum_{n = 0}^\infty e^{-\beta \hbar c \norm{\vetor{k}} \left(n + \frac12\right)} = e^{-\frac12\beta \hbar c \norm{\vetor{k}}} \sum_{n=0}^\infty e^{- \beta \hbar c \norm{\vetor{k}} n} = \frac{e^{-\frac12 \beta \hbar c \norm{\vetor{k}}}}{1 - e^{-\beta\hbar c \norm{\vetor{k}}}} = \frac12 \csch\left(\frac12 \beta \hbar c \norm{\vetor{k}}\right)
    \end{equation*}
    portanto a grande função de partição é dada por
    \begin{equation*}
        \Xi = \prod_{\vetor{k},j} \Xi_{\vetor{k},j} = \prod_{\vetor{k},j} \frac12 \csch\left(\frac12 \beta \hbar c\norm{\vetor{k}}\right) \implies \ln \Xi = - \sum_{\vetor{k}, j} \ln\left[2\sinh\left(\frac12 \beta \hbar c \norm{\vetor{k}}\right)\right],
    \end{equation*}
    e como não há acoplamento com a polarização, temos
    \begin{equation*}
        \ln \Xi = - 2 \sum_{\vetor{k}} \ln\left[2\sinh\left(\frac12 \beta \hbar c \norm{\vetor{k}}\right)\right].
    \end{equation*}

    A energia interna é dada por
    \begin{equation*}
        U = - \diffp{\ln \Xi}{\beta} = \sum_{\vetor{k}} \coth\left(\frac12 \beta \hbar c \norm{\vetor{k}}\right) \hbar c \norm{\vetor{k}} = \frac{V}{(2\pi)^d}\int_{\mathbb{R}^d} \dln{d}k \hbar c \norm{\vetor{k}} \coth\left(\frac12 \beta \hbar c \norm{\vetor{k}}\right),
    \end{equation*}
    onde escrevemos \(V = L^d\) para o volume. Como feito na Lista IV\footnote{Botar link}, temos
    \begin{equation*}
        U = \frac{\pi^{\frac{d}{2}}Vd}{(2\pi)^d \Gamma\left(\frac{d}{2} + 1\right)} \int_0^\infty \dli{k} \hbar c k^d \coth\left(\frac12 \beta \hbar c k\right) = \frac{\pi^{\frac{d}{2}}V d}{(hc)^d \Gamma\left(\frac{d}{2} + 1\right)} \int_0^\infty \dli{\epsilon} \epsilon^d \coth\left(\frac12 \beta \epsilon\right).
    \end{equation*}

\end{proof}
