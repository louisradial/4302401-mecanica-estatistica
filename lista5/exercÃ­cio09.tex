\begin{exercício}{Cadeia linear de osciladores harmônicos}{exercício9}
    Considere uma cadeia linear de \(N\) átomos, cada um com massa \(m\), ligados por molas de constante \(\alpha\). A relação de dispersão para ondas se propagando ao longo da cadeia é dada por \(\omega_{\vetor{k}} = 2 \omega_0 \sin\left(\frac{a \norm{\vetor{k}}}{2}\right)\), em que \(\omega_0 = \sqrt{\frac{\alpha}{m}}\) e \(a \norm{k} \in (-\pi,\pi)\). Mostre que a densidade de estados \(D(\epsilon)\) é dada por
    \begin{equation*}
        D(\epsilon) = \frac{2N}{\pi\sqrt{\epsilon_0^2 - \epsilon^2}},
    \end{equation*}
    em que \(\epsilon_0 = 2\hbar \omega_0\). Mostre também que \(\int_0^{\epsilon_0} \dli{\epsilon} D(\epsilon) = N\), efetuando explicitamente a integral.
\end{exercício}
\begin{proof}[Resolução]

\end{proof}
