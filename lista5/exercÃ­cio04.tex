\begin{exercício}{Temperatura de condensação de Bose-Einstein para bósons massivos}{exercício4}
    Considere um gás ideal de partículas bosônicas, de spin nulo e massa \(m\). Obtenha uma expressão para a temperatura de condensação de Bose-Einstein nas seguintes situações:
    \begin{enumerate}[label=(\alph*)]
        \item supondo que o espectro de energia seja dado por \(\epsilon = \hbar c \norm{\vetor{k}}\); e
        \item supondo que a densidade de estados \(D(\epsilon)\) seja dada por \(\alpha \epsilon^n\) para \(\epsilon > 0\) e nula caso contrário, com \(\alpha > 0\) e \(n > 0\) constantes.
    \end{enumerate}
\end{exercício}
\begin{proof}[Resolução do item (a)]
    O número de partículas na temperatura de condensação é dada por
    \begin{equation*}
        N = \sum_{\vetor{k}} \frac{1}{e^{\beta_0 \epsilon_{\vetor{k}}} - 1} = \frac{4\pi V}{(2\pi)^3} \int_0^\infty \dli{k} \frac{k^2}{e^{\beta_0 \epsilon_{\vetor{k}}} - 1} = \frac{4\pi V}{(2\pi \beta_0 \hbar c)^3} \int_0^\infty \dli{\xi} \frac{\xi^2}{e^{\xi} - 1} = \frac{V \zeta(3)}{\pi^2(\beta_0 \hbar c)^3},
    \end{equation*}
    isto é,
    \begin{equation*}
        T_0 = \frac{\hbar c}{k_B}\left[\frac{\pi^2 N}{\zeta(3)V}\right]^{\frac13}
    \end{equation*}
    é a temperatura de condensação.
\end{proof}
\begin{proof}[Resolução do item (b)]
    O número de partículas é dado por
    \begin{equation*}
        N = V\int_{\mathbb{R}} \dli{\epsilon} \frac{D(\epsilon)}{e^{\beta_0 \epsilon} - 1} = \alpha V \int_{0}^{\infty} \dli{\epsilon}\frac{\epsilon^n}{e^{\beta_0 \epsilon} - 1} = \frac{\alpha V}{\beta^{n+1}} \int_0^\infty \dli{\xi}\frac{\xi^n}{e^{\xi} - 1} = \frac{\alpha V \zeta(n + 1)\Gamma(n+1)}{\beta^{n+1}},
    \end{equation*}
    pelo \cref{lem:integral_bose_einstein}. Assim,
    \begin{equation*}
        T_0 = \frac{1}{k_B}\left[\frac{N}{\alpha \zeta(n+1) \Gamma(n+1) V}\right]^{\frac1{n+1}}
    \end{equation*}
    é a temperatura de condensação. Para \(n = 2\) e \(\alpha = \frac{4\pi}{(h c)^3}\), recuperamos o resultado do item (a).
\end{proof}
