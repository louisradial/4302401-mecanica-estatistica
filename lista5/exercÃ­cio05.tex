\begin{exercício}{Entropia e calor específico a volume constante para um sistema bosônico}{exercício5}
    Considere um sistema de bósons ideais de spin nulo dentro de um recipiente de volume \(V\).
    \begin{enumerate}[label=(\alph*)]
        \item Mostre que a entropia acima da temperatura de condensação \(T_0\) deve ser dada pela expressão
            \begin{equation*}
                S = \frac{k_BV}{\lambda^3}\left[\frac52 g_{\frac52}(z) - \frac{\mu}{k_B T}g_{\frac32}(z)\right],
            \end{equation*}
            onde \(g_{\alpha}(z) = \sum_{n = 1}^\infty \frac{z^n}{n^\alpha}\) é a função polilogarítmica.
        \item Dados \(N\) e \(V\), qual a expressão da entropia abaixo de \(T_0?\) Qual a entropia associada às partículas do condensado?
        \item A partir da expressão para a entropia, mostre que o calor específico a volume constante acima de \(T_0\) é dado pela expressão
            \begin{equation*}
                c_V = \frac{3}{4} k_B \left[5 \frac{g_{\frac52}(z)}{g_{\frac32}(z)} - 3 \frac{g_{\frac32}(z)}{g_{\frac12}(z)}\right].
            \end{equation*}
        \item Mostre que abaixo de \(T_0\), o calor específico a volume constante é dado pela expressão
            \begin{equation*}
                c_V = \frac{15}{4} k_B \frac{v}{\lambda^3}g_{\frac52}(1).
            \end{equation*}
    \end{enumerate}
\end{exercício}
\begin{proof}[Resolução]
    Acima da temperatura de condensação, o grande potencial é dado por
    \begin{align*}
        \Phi = -k_B T \ln\Xi &= k_B T \sum_{\vetor{k}} \ln\left(1 - e^{\beta \mu -\beta \epsilon_{\vetor{k}}}\right)\\
                             &= \frac{4\pi V}{(2\pi)^3\beta} \int_{0}^\infty\dli{k} k^2\ln\left[1 - z e^{-\beta \epsilon_{\vetor{k}}}\right]\\
                             &= \frac{V}{(2\pi)^2 \beta} \left(\frac{2m}{\hbar^2}\right)^{\frac32}\int_0^\infty \dli{\epsilon} \epsilon^{\frac12} \ln\left[1 - z e^{- \beta \epsilon}\right].
    \end{align*}
    Com a expansão \(\ln(1 - x) = - \sum_{n = 1}^\infty \frac{x^n}{n}\), temos
    \begin{align*}
        \Phi &= -\frac{V}{(2\pi)^2 \beta} \left(\frac{2 m}{\hbar^2}\right)^{\frac32} \int_0^\infty \dli{\epsilon} \epsilon^{\frac12} \sum_{n = 1}^\infty \frac{z^n}{n}  e^{- \beta \epsilon n}\\
             &= - \frac{V}{(2\pi)^2 \beta} \left(\frac{2m}{\hbar^2}\right)^{\frac32} \sum_{n = 1}^\infty \frac{z^n}{n} \int_0^\infty \dli{\epsilon} \epsilon^{\frac12} e^{-\beta \epsilon n}\\
             &= - \frac{V}{(2\pi)^2 \beta^{\frac52}} \left(\frac{2m}{\hbar^2}\right)^{\frac32} \sum_{n = 1}^\infty \frac{z^n}{n^{\frac52}} \int_0^\infty \dli{\xi} \xi^{\frac12} e^{-\xi}\\
             &= -\frac{\Gamma\left(\frac32\right)V}{(2\pi)^2 \beta}\left(\frac{2m}{\hbar^2 \beta}\right)^{\frac32} \sum_{n = 1}^{\infty} \frac{z^n}{n^{\frac52}}\\
             &= - \frac{V}{\beta} \left(\frac{2\pi m k_BT}{h^2}\right)^{\frac32} g_{\frac52}(z)\\
             &= -\frac{V k_B T}{\lambda^3} g_{\frac52}(z),
    \end{align*}
    onde \(\lambda = \sqrt{\frac{h^2}{2\pi m k_B T}}\) é o comprimento de onda térmico de de Broglie. Com isso, o número de partículas é dado por
    \begin{equation*}
        N = z\diffp{\ln \Xi}{z}[V, T] = \frac{V}{\lambda^3} z \diffp*{\sum_{n=1}^\infty \frac{z^n}{n^{\frac52}}}{z} = \frac{V}{\lambda^3}\sum_{n = 1}^\infty \frac{z^n}{n^{\frac32}} = \frac{V}{\lambda^3}g_{\frac32}(z)
    \end{equation*}
    a entropia é dada por
    \begin{equation*}
        S = - \diffp{\Phi}{T}[V, N] = - k_B \beta^2 \diffp{\Phi}{\beta}[N, V] = -k_B \beta^2 \left[\diffp{\Phi}{\beta}[z, V] + \diffp{\Phi}{z}[\beta] \diffp{z}{\beta}[N]\right]
    \end{equation*}
\end{proof}
