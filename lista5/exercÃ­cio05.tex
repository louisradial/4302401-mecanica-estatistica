\begin{exercício}{Entropia e calor específico a volume constante para um sistema bosônico}{exercício5}
    Considere um sistema de bósons ideais de spin nulo dentro de um recipiente de volume \(V\).
    \begin{enumerate}[label=(\alph*)]
        \item Mostre que a entropia acima da temperatura de condensação \(T_0\) deve ser dada pela expressão
            \begin{equation*}
                S = \frac{k_BV}{\lambda^3}\left[\frac52 g_{\frac52}(z) - \frac{\mu}{k_B T}g_{\frac32}(z)\right],
            \end{equation*}
            onde \(g_{\alpha}(z) = \sum_{n = 1}^\infty \frac{z^n}{n^\alpha}\) é a função polilogarítmica.
        \item Dados \(N\) e \(V\), qual a expressão da entropia abaixo de \(T_0?\) Qual a entropia associada às partículas do condensado?
        \item A partir da expressão para a entropia, mostre que o calor específico a volume constante acima de \(T_0\) é dado pela expressão
            \begin{equation*}
                c_V = \frac{3}{4} k_B \left[5 \frac{g_{\frac52}(z)}{g_{\frac32}(z)} - 3 \frac{g_{\frac32}(z)}{g_{\frac12}(z)}\right].
            \end{equation*}
        \item Mostre que abaixo de \(T_0\), o calor específico a volume constante é dado pela expressão
            \begin{equation*}
                c_V = \frac{15}{4} k_B \frac{v}{\lambda^3}g_{\frac52}(1).
            \end{equation*}
    \end{enumerate}
\end{exercício}
\begin{proof}[Resolução]

\end{proof}
