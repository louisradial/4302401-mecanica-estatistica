\begin{exercício}{Calor específico de gás de excitações elementares}{exercício3}
    Com o resultado obtido no \cref{ex:exercício2}, escreva a expressão para o calor específico a volume constante no regime de baixas temperaturas para uma, duas e três dimensões e \(n = 1\) (fônons e fótons) e \(n = 2\) (mágnons ferromagnéticos).
\end{exercício}
\begin{proof}[Resolução]
    Em uma dimensão temos
    \begin{equation*}
        c_V^\mathrm{f} = \frac{\pi^2 k_BT}{3T_D}\quad\text{e}\quad
        c_V^\mathrm{m} = \frac{3 \sqrt{\pi} \zeta\left(\frac32\right)k_B}{8} \sqrt{\frac{T}{T_D}},
    \end{equation*}
    em duas dimensões temos
    \begin{equation*}
        c_V^\mathrm{f} = 24 \zeta(3) k_B\left(\frac{T}{T_D}\right)^{2}\quad\text{e}\quad
        c_V^\mathrm{m} = \frac{2\pi^2k_B T}{3T_D},
    \end{equation*}
    e em três dimensões temos
    \begin{equation*}
        c_V^\mathrm{f} = \frac{12 \pi^4 k_B}{5}\left(\frac{T}{T_D}\right)^3\quad\text{e}\quad
        c_V^\mathrm{m} = \frac{135\sqrt{\pi} \zeta\left(\frac{5}{2}\right)}{16}\left(\frac{T}{T_D}\right)^{\frac32},
    \end{equation*}
    como os calores específicos a volume constante para fônons e para mágnons ferromagnéticos.
\end{proof}
