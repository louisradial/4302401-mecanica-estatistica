\begin{exercício}{Cadeia linear de osciladores harmônicos}{exercício9}
    Considere uma cadeia linear de \(N\) átomos, cada um com massa \(m\), ligados por molas de constante \(\alpha\). A relação de dispersão para ondas se propagando ao longo da cadeia é dada por \(\omega_{\vetor{k}} = 2 \omega_0 \sin\left(\frac{a \norm{\vetor{k}}}{2}\right)\), em que \(\omega_0 = \sqrt{\frac{\alpha}{m}}\) e \(a \norm{k} \in (-\pi,\pi)\). Mostre que a densidade de estados \(D(\epsilon)\) é dada por
    \begin{equation*}
        D(\epsilon) = \frac{2N}{\pi\sqrt{\epsilon_0^2 - \epsilon^2}},
    \end{equation*}
    em que \(\epsilon_0 = 2\hbar \omega_0\). Mostre também que \(\int_0^{\epsilon_0} \dli{\epsilon} D(\epsilon) = N\), efetuando explicitamente a integral.
\end{exercício}
\begin{proof}[Resolução]
    A função de partição é dada por
    \begin{equation*}
        \ln \Xi = - \sum_{\vetor{k}} \ln\left(1 - e^{-\beta \hbar \omega_{\vetor{k}}}\right),
    \end{equation*}
    portanto a energia interna é
    \begin{equation*}
        U = -\diffp{\ln{\Xi}}{\beta} = \sum_{\vetor{k}} \frac{\hbar \omega_{\vetor{k}}}{e^{\beta \hbar \omega_{\vetor{k}}} - 1}.
    \end{equation*}
    No limite termodinâmico temos
    \begin{equation*}
        U = \frac{L}{\pi} \int_{0}^{\frac{\pi}{a}}\dli{k} \frac{2\hbar \omega_0 \sin\left(\frac{a k}{2}\right)}{\exp\left[2 \beta \hbar \omega_0 \sin\left(\frac{a k}{2}\right)\right] - 1}
    \end{equation*}
    portanto com a mudança de variáveis \(\epsilon = 2 \hbar \omega_0 \sin\left(\frac{ak}{2}\right)\), temos \(\dl{\epsilon} = a \hbar \omega_0 \cos\left(\frac{a k}{2}\right) \dl{k}\) e então
    \begin{align*}
        U &= \frac{L}{a \hbar \omega_0 \pi} \int_0^{\epsilon_0} \dli{\epsilon} \frac{\epsilon}{\sqrt{1 - \left(\frac{\epsilon}{2 \hbar \omega_0}\right)^2}(e^{\beta \epsilon} - 1)}\\
          &= \int_0^{\epsilon_0} \frac{2L \epsilon}{\pi a \sqrt{\epsilon_0^2 - \epsilon^2}} f(\epsilon),
    \end{align*}
    onde \(\epsilon_0 = 2 \hbar \omega_0\) e \(f(\epsilon) = \frac{1}{\exp(\beta \epsilon) - 1}\) é a distribuição de Bose-Einstein. Isto é,
    \begin{equation*}
        D(\epsilon) = \frac{2L}{\pi a \sqrt{\epsilon_0^2-\epsilon^2}}
    \end{equation*}
    é a densidade de estados. O número de osciladores é dado por
    \begin{equation*}
        N = \frac{L}{\pi} \int_0^{\frac{\pi}{a}} \dli{k} = \frac{L}{a},
    \end{equation*}
    portanto obtemos
    \begin{equation*}
        D(\epsilon) = \frac{2N}{\pi \sqrt{\epsilon_0^2 - \epsilon^2}}.
    \end{equation*}
    Notemos que
    \begin{equation*}
        \int_0^{\epsilon_0} \dli{\epsilon} D(\epsilon) = \frac{2N}{\pi}\int_0^{\epsilon_0} \dli{\epsilon} \left(\epsilon_0^2 - \epsilon^2\right)^{-\frac12} = \frac{2N}{\pi} \int_0^{\frac{\pi}{2}} \epsilon_0 \cos\theta\dli{\theta} \left(\epsilon_0^2 - \epsilon_0^2 \sin^2 \theta\right)^{-\frac12} = N,
    \end{equation*}
    como desejado.
\end{proof}
