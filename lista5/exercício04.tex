\begin{exercício}{Temperatura de condensação de Bose-Einstein para bósons massivos}{exercício4}
    Considere um gás ideal de partículas bosônicas, de spin nulo e massa \(m\). Obtenha uma expressão para a temperatura de condensação de Bose-Einstein nas seguintes situações:
    \begin{enumerate}[label=(\alph*)]
        \item supondo que o espectro de energia seja dado por \(\epsilon = \hbar c \norm{\vetor{k}}\); e
        \item supondo que a densidade de estados \(D(\epsilon)\) seja dada por \(\alpha \epsilon^2\) para \(\epsilon > 0\) e nula caso contrário, com \(\alpha > 0\) constante.
    \end{enumerate}
\end{exercício}
\begin{proof}[Resolução]

\end{proof}
