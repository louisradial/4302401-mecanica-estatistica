\begin{exercício}{Capacidade térmica para o sólido de Einstein}{exercício7}
    A teoria de Einstein para a capacidade térmica de sólidos corresponde a uma coleção de \(3N\) osciladores harmônicos de mesma frequência \(\omega_E\), o que significa que a densidade de estados \(D(\epsilon) = 3 N \delta(E - \hbar \omega_E)\). Ache a capacidade térmica \(C\) segundo essa teoria e faça um esboço de \(C\) versus \(T\).
\end{exercício}
\begin{proof}[Resolução]
    Segundo essa teoria, a energia interna é dada por
    \begin{equation*}
        U = V \int_{\mathbb{R}} \dli{\epsilon} \frac{\epsilon D(\epsilon)}{e^{{\beta \epsilon}} - 1} = 3NV \int_{\mathbb{R}} \dli{\epsilon} \frac{\epsilon \delta(\epsilon - \hbar \omega)}{e^{\beta \epsilon} - 1} = 3 NV \int_{\mathbb{R}} \frac{\hbar \omega \delta(\epsilon - \hbar \omega)}{e^{\beta \hbar \omega} - 1} = \frac{3 NV \hbar \omega}{e^{\beta \hbar \omega} - 1},
    \end{equation*}
    onde usamos a propriedade \(\delta(x - a)f(x) = \delta(x - a)f(a)\). Com isso, a capacidade térmica é dada por
    \begin{equation*}
        C_V = \diffp{U}{T}[N,V] = -\frac{3 NV (\hbar \omega)^2 e^{\beta \hbar \omega}}{(e^{\beta \hbar \omega} - 1)^2} \diffp{\beta}{T} = 3NV k_B (\beta \hbar \omega)^2 \frac{e^{\beta \hbar \omega}}{(e^{\beta \hbar \omega} - 1)^2} = 3 NV k_B \frac{\left(\frac{\hbar \omega}{k_B T}\right)^2\exp\left(\frac{\hbar \omega}{k_B T}\right)}{\left[\exp\left(\frac{\hbar \omega}{k_B T}\right)-1\right]^2}.
    \end{equation*}
    \begin{figure}[!ht]
         \centering
        \begin{tikzpicture}
            \def\b{3/2};
            \def\n{3/2}
            \begin{axis}[
                width=0.95\linewidth,
                height=0.20\textheight,
                xmin=0, xmax=5.5,
                ymin=0,ymax=1.2,
                domain=0:5.5,
                samples=450,
                axis lines=middle,
                xlabel={\(T\)},
                xlabel style = {anchor=north east},
                % ylabel near ticks,
                ylabel={\(C_V\)},
                legend pos=south east,
                ytick={0,0.5,1},
                xtick={0,1,2,3,4,5},
                smooth,
                grid,
                xticklabels={0, \(\frac{\hbar \omega}{k_B}\), \(\frac{2\hbar \omega}{k_B}\), \(\frac{3\hbar \omega}{k_B}\), \(\frac{4\hbar \omega}{k_B}\), \(\frac{5\hbar \omega}{k_B}\)},
                yticklabels={0, \(\frac32 NV k_B\), \(3 NV k_B\)},
                ]
                \addplot[thick, Mauve] {((1/x)^2 * exp(1/x))/((exp(1/x) - 1)^2)};
                % \addlegendentry{\(C_V\)};
                % \addplot[thick, Pink] {(-1 + \b*(sqrt(1 - \n^2 * (sin(deg(x))^2))/cos(deg(x))))/(1 + \b*(sqrt(1 - \n^2 * (sin(deg(x))^2))/cos(deg(x))))};
                % \addlegendentry{\(\tilde{E}^0_T/\tilde{E}_I^0\)};
                % \addplot[Pink, dotted] {1};
            \end{axis}
        \end{tikzpicture}
        \caption{Capacidade térmica para o sólido de Einstein.}
    \end{figure}

    Invertendo a série para \(\frac{e^x - 1}{x}\) ao redor de \(x = 0\), obtemos
    \begin{equation*}
        \frac{x}{e^x - 1} = 1 - \frac12 x + \frac1{12} x^2 + O(x^3)
    \end{equation*}
    portanto
    \begin{equation*}
        \frac{x^2}{(e^x - 1)^2} = 1 - x + \frac{5}{12} x^2 + O(x^3)
    \end{equation*}
    e então
    \begin{equation*}
        \frac{x^2 e^x}{(e^x - 1)^2} = 1 - \frac1{12} x^2 + O(x^3),
    \end{equation*}
    isto é, para altas temperaturas temos a expressão assintótica
    \begin{equation*}
        C_V \simeq 3NV k_B \left[1 - \frac{1}{12}\left(\frac{\hbar \omega}{k_B T}\right)^2\right]
    \end{equation*}
    em até segunda ordem\footnote{Em verdade, em até terceira ordem.} de \(\frac{\hbar \omega}{k_B T}\).
\end{proof}
