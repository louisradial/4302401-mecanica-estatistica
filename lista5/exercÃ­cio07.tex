\begin{exercício}{Capacidade térmica para o sólido de Einstein}{exercício7}
    A teoria de Einstein para a capacidade térmica de sólidos corresponde a uma coleção de \(3N\) osciladores harmônicos de mesma frequência \(\omega_E\), o que significa que a densidade de estados \(D(\epsilon) = 3 N \delta(E - \hbar \omega_E)\). Ache a capacidade térmica \(C\) segundo essa teoria e faça um esboço de \(C\) versus \(T\).
\end{exercício}
\begin{proof}[Resolução]

\end{proof}
