\begin{exercício}{Calor específico de gás de excitações elementares}{exercício2}
    Considere um modelo para um sistema \(d\)-dimensional, com hamiltoniana diagonalizada por uma transformação de Fourier. As excitações elementares estão relacionadas com o operador \(\mathcal{H} = \sum_{\vetor{k}} \epsilon_{\vetor{k}} a_{\vetor{k}}^\dag a_{\vetor{k}},\) onde a soma é efetuada sobre a primeira zona de Brillouin e os operadores de criação e aniquilação obedecem as regras de comutação de Bose-Einstein. O espectro de energia é dado por \(\epsilon_{\vetor{k}} = A\norm{k}^n\), sendo \(A\) uma constante positiva e \(n \in \mathbb{N}\), por exemplo \(n = 1\) para fótons e \(n = 2\) para mágnons ferromagnéticos. Obtenha uma expressão para o calor específico a volume constante no regime de baixas temperaturas.
\end{exercício}
\begin{proof}[Resolução]

\end{proof}
