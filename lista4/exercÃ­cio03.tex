\begin{exercício}{Gás de elétrons livres ultrarelativísticos}{exercício03}
    Considere um sistema de \(N\) elétrons livres, dentro de uma região de volume \(V\), num regime ultrarelativístico. O espectro de energia é dado por
    \begin{equation*}
        \epsilon = \sqrt{p^2 c^2 + m^2 c^4} \sim pc,
    \end{equation*}
    onde \(\vetor{p}\) é o momento linear.
    \begin{enumerate}[label=(\alph*)]
        \item Calcule a energia de Fermi desse sistema.
        \item Determine a energia do sistema no estado fundamental.
        \item Obtenha uma forma assintótica para o calor específico a volume constante no limite \(T \ll T_F\).
    \end{enumerate}
\end{exercício}
