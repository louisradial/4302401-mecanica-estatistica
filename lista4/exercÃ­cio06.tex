\begin{exercício}{Capacidade térmica de um gás de elétrons livres degenerados}{exercício06}
    Use a identidade termodinâmica
    \begin{equation*}
        \diffp{U}{T}[N] = \diffp{U}{T}[\mu] - \diffp{U}{\mu}[T] \frac{\diffp{N}{T}[\mu]}{\diffp{N}{\mu}[T]}
    \end{equation*}
    para mostrar que a capacidade térmica a volume constante \(C_v\) de um gás de elétrons livres degenerados vale
    \begin{equation*}
        C_v = \frac{\pi^2}{3} D(\epsilon_F) k_B^2T,
    \end{equation*}
    onde \(D(\epsilon)\) é a densidade de orbitais. A partir da expressão acima, reobtenha a expressão para \(C_v = \frac{\pi^2}{2} N k_B \frac{T}{T_F}\), que fora deduzida em sala de aula.
\end{exercício}
\begin{proof}[Resolução]
    Do \cref{ex:exercício04}, temos
    \begin{equation*}
        U = \frac{2 \gamma V \mu^{\frac{5}{2}}}{5(2\pi)^2 \left(\frac{\hbar^2}{2m}\right)^{\frac{3}{2}}} \left[1 + \frac{5\pi^2}{8}\left(\frac{T}{T_F}\right)^2 + \dots\right],
    \end{equation*}
    \begin{equation*}
        N = \frac{2\gamma V \mu^{\frac{3}{2}}}{3(2\pi)^2 \left(\frac{\hbar^2}{2m}\right)^{\frac{3}{2}}}\left[1 + \frac{\pi^2}{8}\left(\frac{T}{T_F}\right)^2 + \dots\right],
    \end{equation*}
    \begin{equation*}
        \mu = \epsilon_F \left[1 - \frac{\pi^2}{12}\left(\frac{T}{T_F}\right)^2 + \dots\right],
    \end{equation*}
    e
    \begin{equation*}
        D(\epsilon) = \frac{1}{(2\pi)^2 \left(\frac{\hbar^2}{2m}\right)^{\frac{3}{2}}} \epsilon^{\frac12},
    \end{equation*}
    onde \(k_B T_F = \epsilon_F\) é a energia de Fermi.
\end{proof}
