\begin{exercício}{Entropia de um gás ideal quântico}{exercício01}
    Mostre que a entropia de um gás ideal quântico pode ser escrita da seguinte forma
    \begin{equation*}
        S = - k_B \sum_{i} \left[f_i \ln f_i \pm (1 \mp f_i)\ln(1 \mp f_i)\right],
    \end{equation*}
    onde os sinais superiores (inferiores) referem-se a férmions (bósons) e
    \begin{equation*}
        f_i = \mean{n_i} = \frac1{\exp\left(\beta \epsilon_i - \beta \mu\right) \pm 1}
    \end{equation*}
    é a distribuição de Fermi-Dirac (Bose-Einstein). Mostre que os resultados acima também podem ser utilizados para o gás ideal clássico.
\end{exercício}
