\begin{exercício}{Potencial químico de um gás clássico}{exercício5}
    Mostre que o potencial químico de um gás clássico de \(N\) partículas monoatômicas no volume \(V\), a temperatura \(T\), pode ser escrito na forma
    \begin{equation*}
        \mu = k_B T \ln\left(\frac{\lambda^3}{v}\right),
    \end{equation*}
    onde \(v = \frac{V}{N}\) e \(\lambda = \frac{h}{\sqrt{2 \pi mk_B T}}\) é o comprimento de onda térmico. Obtenha agora a primeira correção quântica desse resultado. Isto é, mostre que
    \begin{equation*}
        \mu - k_BT \ln\left(\frac{\lambda^3}{v}\right) = A\left(\frac{\lambda^3}{v}\right) + \dots
    \end{equation*}
    e obtenha explicitamente o prefator \(A\) nos casos de férmions e bósons.
\end{exercício}
