\begin{exercício}{Entropia de um gás ideal quântico}{exercício01}
    Mostre que a entropia de um gás ideal quântico pode ser escrita da seguinte forma
    \begin{equation*}
        S = - k_B \sum_{i} \left[f_i \ln f_i \pm (1 \mp f_i)\ln(1 \mp f_i)\right],
    \end{equation*}
    onde os sinais superiores (inferiores) referem-se a férmions (bósons) e
    \begin{equation*}
        f_i = \mean{n_i} = \frac1{\exp\left(\beta \epsilon_i - \beta \mu\right) \pm 1}
    \end{equation*}
    é a distribuição de Fermi-Dirac (Bose-Einstein). Mostre que os resultados acima também podem ser utilizados para o gás ideal clássico.
\end{exercício}
\begin{proof}[Resolução]
    Notemos que
    \begin{equation*}
        f_j = \frac{\exp(\beta \mu - \beta \epsilon_j)}{1 \pm \exp(\beta \mu - \beta \epsilon_j)} \implies \exp(\beta \mu - \beta \epsilon_j) = \frac{f_j}{1 \mp f_j} \implies 1 \pm \exp(\beta \mu - \beta \epsilon_j) = \frac{1}{1 \mp f_j},
    \end{equation*}
    portanto a grande função de partição é dada por
    \begin{equation*}
        \ln{\Xi} = \pm \sum_j \ln[1 \pm \exp(\beta \mu - \beta \epsilon_j)] = \pm \sum_j \ln\left(\frac{1}{1 \mp f_j}\right) = \mp \sum_j \ln(1 \mp f_j).
    \end{equation*}
    Como a entropia é dada por \(S = -\diffp{\Phi}{T}[V,\mu]\), temos
    \begin{align*}
        S = k_B \beta^2 \diffp*{(-\beta^{-1} \ln\Xi)}{\beta}[T,\mu]
          &= k_B \ln\Xi - k_B \beta \diffp*{\ln{\Xi}}{\beta} \\
          &= \mp k_B  \sum_i \ln(1 \mp f_i) \mp k_B \beta \sum_{j}\diffp*{\ln\left(\frac{1}{1 \mp f_j}\right)}{\beta}\\
          &= \mp k_B \sum_j\left[ \ln(1 \mp f_j) \pm (\beta\mu - \beta \epsilon_j) \exp(\beta \mu - \beta \epsilon_j) (1 \mp f_j) \right]\\
          &= \mp k_B \sum_j\left[\ln(1\mp f_j) \pm f_j \ln\left(\frac{f_j}{1 \mp f_j}\right)\right]\\
          &= -k_B\sum_j\left[f_j \ln f_j \pm (1 \mp f_j)\ln(1 \mp f_j) \right]
    \end{align*}
    como desejado.

    No limite clássico, temos \(\exp(\beta \epsilon_j - \beta \mu) \gg 1\), de modo que \(f_j \simeq \exp(\beta \mu - \beta \epsilon_j)\) e \(\ln(1 \mp f_j) \simeq \mp f_j\). Então temos
    \begin{equation*}
        -k_B \sum_i\left[f_i \ln f_i \pm (1 \mp f_i)\ln(1 \mp f_i)\right] = -k_B \sum_j \left[(\beta \mu - \beta \epsilon_j) \exp(\beta \mu -\beta \epsilon_j) - \exp(\beta \mu - \beta \epsilon_j)\right].
    \end{equation*}
    em primeira ordem em \(f_j\). Neste limite a grande função de partição é dada por
    \begin{equation*}
        \ln\Xi = \sum_j \exp(\beta \mu - \beta \epsilon_j),
    \end{equation*}
    portanto a entropia é dada por
    \begin{align*}
        S = \diffp*{(k_B T \ln\Xi)}{T}[V, \mu] &= k_B \sum_j \exp(\beta \mu - \beta \epsilon_j) + k_B T \sum_j (\mu - \epsilon_j)\exp(\beta \mu - \beta \epsilon_j) \diffp{\beta}{T}\\
                                               &= k_B \sum_j \exp(\beta \mu - \beta \epsilon_j) - k_B \sum_j (\beta \mu - \beta \epsilon_j) \exp(\beta \mu - \beta \epsilon_j)\\
                                               &= -k_B\sum_j\left[(\beta \mu - \beta \epsilon_j)\exp(\beta \mu - \beta \epsilon_j) - \exp(\beta \mu - \beta \epsilon_j)\right]\\
                                               &= -k_b \sum_{j} \left[f_j \ln f_j \pm (1 \mp f_j)\ln(1 \mp f_j)\right],
    \end{align*}
    como desejado.
\end{proof}
