\begin{exercício}{Capacidade térmica de um gás de elétrons livres degenerados}{exercício06}
    Use a identidade termodinâmica
    \begin{equation*}
        \diffp{U}{T}[N] = \diffp{U}{T}[\mu] - \diffp{U}{\mu}[T] \frac{\diffp{N}{T}[\mu]}{\diffp{N}{\mu}[T]}
    \end{equation*}
    para mostrar que a capacidade térmica a volume constante \(C_v\) de um gás de elétrons livres degenerados vale
    \begin{equation*}
        C_v = \frac{\pi^2}{3} D(\epsilon_F) k_B^2T,
    \end{equation*}
    onde \(D(\epsilon)\) é a densidade de orbitais. A partir da expressão acima, reobtenha a expressão para \(C_v = \frac{\pi^2}{2} N k_B \frac{T}{T_F}\), que fora deduzida em sala de aula.
\end{exercício}
