\begin{exercício}{Sistema de osciladores quânticos localizados}{exercício6}
    Um sistema de \(N\) osciladores quânticos localizados e independentes está em contato com um reservatório térmico a temperatura \(T\). Os níveis de energia de cada oscilador são dados por \(\epsilon_n = \hbar \omega_0 (n + \frac12)\), com \(n\) ímpar.
    \begin{enumerate}[label=(\alph*)]
        \item Obtenha uma expressão para a energia interna \(u\) por oscilador em função da temperatura \(T\). Esboce um gráfico de \(u\) contra \(T\). Qual a expressão no limite clássico, isto é, para \(\hbar \omega_0 \ll k_B T\)?
        \item Obtenha uma expressão para a entropia por oscilador em função da temperatura. Esboce um gráfico da entropia contra a temperatura. Qual a expressão da entropia no limite clássico?
        \item  Qual a expressão do calor específico no limite clássico?
    \end{enumerate}
\end{exercício}
\begin{proof}[Resolução]
    Como os osciladores são não interagentes, a função de partição é
    \begin{align*}
        Z &= \left(\sum_{n = 1}^\infty e^{-\beta\epsilon_{2n-1}}\right)^N = \left[\sum_{n=1}^\infty e^{-\beta \hbar \omega_0\left(2n - \frac12\right)}\right]^N = \left[e^{\frac12\beta \hbar \omega_0} \sum_{n=1}^\infty e^{-2\beta \hbar \omega_0 n}\right]^N = \left(\frac{e^{- 2\beta \hbar \omega_0}}{1 - e^{-2\beta \hbar \omega_0}} e^{\frac12 \beta \hbar \omega_0}\right)^N\\
          &= \left(\frac{e^{-\frac12\beta \hbar \omega_0}}{e^{\beta \hbar \omega_0} - e^{-\beta \hbar \omega_0}}\right)^N = \left[2e^{\frac12 \beta \hbar \omega_0}\sinh\left(\beta\hbar \omega_0\right)\right]^{-N},
    \end{align*}
    portanto a energia livre de Helmholtz por oscilador é dada por
    \begin{align*}
        f &= -\frac{\ln Z}{N \beta} = \frac{1}{\beta} \ln \left[2 e^{\frac12 \beta \hbar \omega_0}\sinh\left(\frac12\beta \hbar \omega_0\right)\right]\\
          &=  \frac 12 \hbar \omega_0 + \frac1\beta \ln\left[2\sinh(\beta \hbar \omega_0)\right].
    \end{align*}

    A energia interna por oscilador é
    \begin{align*}
        u &= -\frac1N \diffp{\ln Z}{\beta} = \diffp{(\beta f)}{\beta}= \hbar\omega_0\left[\tanh(\beta \hbar \omega_0)+\frac12 \right]\\
          &= \hbar \omega_0 \left[\coth\left(\frac{\hbar \omega_0}{k_B T}\right) + \frac12\right].
    \end{align*}
    Para \(\hbar \omega_0 \ll k_B T\), temos
    \begin{align*}
        \coth\left(\frac{\hbar \omega_0}{k_B T}\right) = \frac{1 + e^{-2\left(\frac{\hbar \omega_0}{k_B T}\right)}}{1 - e^{-2\left(\frac{\hbar \omega_0}{k_B T}\right)}}
                                                                 &= 1 + \frac2{e^{2\left(\frac{\hbar \omega_0}{k_B T}\right)} - 1}\\
                                                                 &= 1 + \frac2{\sum_{k = 1}^\infty \left(2\frac{\hbar \omega_0}{k_B T}\right)^2}\\
                                                                 &\simeq \frac{k_B T}{\hbar \omega_0} + \frac{\hbar \omega_0}{3k_B T},
    \end{align*}
    portanto
    \begin{equation*}
        u \simeq u_\mathrm{c} = \hbar \omega_0 \left(\frac{k_B T}{\hbar\omega_0} + \frac{1}{2} + \frac{\hbar \omega_0}{3 k_B T}\right)
    \end{equation*}
    é a expressão para a energia interna no limite clássico.

    \begin{figure}[!ht]
        \centering
        \begin{tikzpicture}
            \begin{axis}[
                width=0.8\linewidth,
                height=0.25\textheight,
                xmin=0, xmax=5.15,
                ymin=0,ymax=5.75,
                domain=0:10.5,
                samples=500,
                axis lines=middle,
                xlabel={T},
                xlabel near ticks,
                ylabel near ticks,
                ylabel={\(u\)},
                legend pos=south east,
                ytick={3/2},
                xtick={0, 1, 2, 3, 4,5},
                xticklabels={0, \(\frac{\hbar \omega_0}{k_B}\), \(\frac{2\hbar \omega_0}{k_B}\), \(\frac{3\hbar \omega_0}{k_B}\), \(\frac{4\hbar \omega_0}{k_B}\), \(\frac{5\hbar \omega_0}{k_B}\)},
                yticklabels={\(\frac32\hbar \omega_0\)},
            ]
            \addplot[thick, Mauve] {1/tanh(1/x) + 1/2};
            \addlegendentry{\(u\)}
            \addplot[thick, dashed, Pink] {x + 1/2 + 1/(3*x)};
            \addlegendentry{\(u_\mathrm{c}\)}
            % \addplot[thin, dashed] {3/2};
            \end{axis}
        \end{tikzpicture}
        \caption{Energia interna por oscilador.}
    \end{figure}

    A entropia por oscilador é dada por
    \begin{align*}
        s = \frac{u - f}{T} = k_B\left\{\frac{\hbar \omega_0}{k_BT}\coth\left(\frac{\hbar\omega_0}{k_BT}\right)- \ln\left[2\sinh\left(\frac{\hbar \omega_0}{k_B T}\right)\right]\right\}.
    \end{align*}
    Para \(\hbar \omega_0 \ll k_B T\), temos
    \begin{align*}
        \ln\left[2\sinh\left(\frac{\hbar \omega_0}{k_B T}\right)\right] &= \ln\left[2 \sum_{k=0}^\infty \frac{\left(\frac{\hbar\omega_0}{k_BT}\right)^{2k+1}}{(2k+1)!}\right]\\
                                                                        &= \ln\left[2\left(\frac{\hbar\omega_0}{k_BT}\right)\sum_{k=0}^\infty \frac{\left(\frac{\hbar\omega_0}{k_BT}\right)^{2k}}{(2k+1)!}\right]\\
                                                                        &= \ln\left(\frac{2\hbar\omega_0}{k_BT}\right) + \ln\left[1 + \sum_{k=1}^\infty\frac{\left(\frac{\hbar\omega_0}{k_BT}\right)^{2k}}{(2k+1)!}\right]\\
                                                                        &\simeq \ln\left(\frac{2\hbar\omega_0}{k_BT}\right) + \frac16\left(\frac{\hbar\omega_0}{k_BT}\right)^2,
    \end{align*}
    portanto
    \begin{align*}
        s \simeq s_\mathrm{c} &= \frac{\hbar \omega_0}{T} \left(\frac{k_B T}{\hbar\omega_0} + \frac{1}{2} + \frac{\hbar \omega_0}{3 k_B T}\right) - k_B\ln\left(\frac{2\hbar\omega_0}{k_BT}\right) - \frac16\left(\frac{\hbar\omega_0}{k_BT}\right)^2\\
                              &= k_B \left[1 + \frac{\hbar\omega_0}{2k_B T} + \frac16\left(\frac{\hbar \omega_0}{k_B T}\right)^2 - \ln\left(\frac{2\hbar \omega_0}{k_B T}\right)\right]
    \end{align*}
    é a expressão para a entropia no limite clássico.

    \begin{figure}[!ht]
        \centering
        \begin{tikzpicture}
            \begin{axis}[
                width=0.8\linewidth,
                height=0.25\textheight,
                xmin=0, xmax=20.5,
                ymin=0,ymax=3.75,
                domain=0:70.5,
                samples=500,
                axis lines=middle,
                xlabel={T},
                xlabel near ticks,
                ylabel near ticks,
                ylabel={\(s\)},
                legend pos=south east,
                ytick={1},
                xtick={0, 4, 8, 12, 16,20},
                xticklabels={0, \(\frac{4\hbar \omega}{k_B}\), \(\frac{8\hbar \omega_0}{k_B}\), \(\frac{12\hbar \omega_0}{k_B}\), \(\frac{16\hbar \omega_0}{k_B}\), \(\frac{20\hbar \omega_0}{k_B}\)},
                yticklabels={\(k_B\)},
            ]
            \addplot[thick, Mauve] {(1/x)/tanh(1/x) - ln(2*sinh(1/x))};
            \addlegendentry{\(s\)}
            \addplot[thick, Pink, dashed] {1 + 1/(2*x) + 1/6*(1/x)^2 - ln(2/x)};
            \addlegendentry{\(s_\mathrm{c}\)}
            \end{axis}
        \end{tikzpicture}
        \caption{Entropia por oscilador.}
    \end{figure}

    \begin{figure}[!ht]
        \centering
        \begin{tikzpicture}
            \begin{axis}[
                width=0.8\linewidth,
                height=0.25\textheight,
                xmin=0, xmax=10.5,
                ymin=0,ymax=1.15,
                domain=0:70.5,
                samples=1000,
                axis lines=middle,
                xlabel={T},
                xlabel near ticks,
                ylabel near ticks,
                ylabel={\(c\)},
                legend pos=south east,
                ytick={1},
                xtick={0, 2, 4, 6, 8,10},
                xticklabels={0, \(\frac{2\hbar \omega_0}{k_B}\), \(\frac{4\hbar \omega_0}{k_B}\), \(\frac{6\hbar \omega_0}{k_B}\), \(\frac{8\hbar \omega_0}{k_B}\), \(\frac{10\hbar \omega_0}{k_B}\)},
                yticklabels={\(k_B\)},
                unbounded coords=jump
            ]
            \addplot[thick, Mauve] {((1/x)/(sinh(1/x)))^2};
            \addlegendentry{\(c\)}
            \addplot[thick, Pink, dashed] {1 - 1/(3*x^2)};
            \addlegendentry{\(c_\mathrm{c}\)}
            \end{axis}
        \end{tikzpicture}
        \caption{Calor específico.}
    \end{figure}

    O calor específico é dado por
    \begin{equation*}
        c = \diffp{u}{T} = \hbar \omega_0 \frac{\frac{\hbar\omega_0}{k_BT^2}\sech^2\left(\frac{\hbar \omega_0}{k_B T}\right)}{\tanh^2\left(\frac{\hbar\omega_0}{k_B T}\right)} = k_B\left[\frac{\hbar \omega_0}{k_B T\sinh\left(\frac{\hbar \omega_0}{k_BT}\right)}\right]^2.
    \end{equation*}
    Para \(\hbar \omega_0\ll k_BT\), temos
    \begin{equation*}
        \left[\frac{\left(\frac{\hbar\omega_0}{k_BT}\right)}{\sinh{\left(\frac{\hbar\omega_0}{k_BT}\right)}}\right]^2 = \left[\frac{1}{\sum_{k=0} \frac{\left(\frac{\hbar\omega_0}{k_BT}\right)^{2k}}{(2k+1)!}}\right]^2 \simeq \left[1 - \frac16\left(\frac{\hbar\omega_0}{k_BT}\right)^2\right]^2 \simeq 1 - \frac13\left(\frac{\hbar\omega_0}{k_BT}\right)^2,
    \end{equation*}
    portanto
    \begin{equation*}
        c\simeq c_\mathrm{c} = k_B\left[1 - \frac13\left(\frac{\hbar \omega_0}{k_B T}\right)^2\right]
    \end{equation*}
    é a expressão para entropia no limite clássico.
\end{proof}
