\begin{exercício}{Gás de rede}{exercício2}
    Considere um gás de rede constituído por \(N\) partículas distribuídas em \(V\) células, com \(V > N\). Suponha que cada célula possa estar vazia ou ocupada por uma única partícula.
    \begin{enumerate}[label=(\alph*)]
        \item Ache a expressão para a densidade de estados \(\Omega(N, V)\) e a entropia por partícula \(s(v)\), onde \(v = \frac{V}{N}\).
        \item Obtenha a pressão \(p\) e escreva uma expressão em termos da densidade \(\rho = \frac1v\). Mostre que o primeiro termo corresponde à lei de Boyle dos gases ideais.
        \item Obtenha uma expressão para o potencial químico absoluto \(\mu\) em função da densidade. Qual é o comportamento do potencial químico nos limites \(\rho \to 0\) e \(\rho \to 1\)?
    \end{enumerate}
\end{exercício}
\begin{proof}[Resolução]
    Dado \(V\) e \(N < V\), temos \(\binom{V}{N}\) maneiras de distribuir as partículas nas células, isto é,
    \begin{equation*}
        \Omega(N, V) = \frac{V!}{(V-N)!N!}
    \end{equation*}
    é a densidades de estados microscópicos. Assim, temos
    \begin{align*}
        S(N, V) &= k_B\left\{\ln(V!) - \ln[(V-N)!] - ln(N!)\right\}\\
                &\simeq k_B\left[\xlnx{V} - \xlnx{(V-N)} - \xlnx{N}\right]\\
                &= k_B \left[V \ln\left(\frac{V}{V-N}\right) + N \ln\left(\frac{V - N}{N}\right)\right]\\
                &= Nk_B \left[v \ln\left(\frac{v}{v - 1}\right) + \ln(v-1)\right]\\
                &= N k_B \left[\xlnx{v} - \xlnx{(v - 1)}\right],
    \end{align*}
    logo a entropia por partícula é
    \begin{equation*}
        s_N(v) = k_B\left[\xlnx{v} - \xlnx{(v - 1)}\right].
    \end{equation*}

    A pressão é dada por
    \begin{equation*}
        \frac{P}{T} = \diffp{s_N}{v}[N] = k_B \ln\left(\frac{v}{v - 1}\right) = -k_B \ln({1 - \rho}),
    \end{equation*}
    onde \(\rho = \frac{N}{V}\) é a densidade de partículas. Como \(\ln(1+x) \simeq x + O(x^2)\), temos
    \begin{equation*}
        \frac{P}{T} \simeq k_B \rho + O(\rho^2) \implies P V \simeq k_B N T
    \end{equation*}
    no limite em que \(\rho \to 0\), correspondente à equação de gases ideais.

    Notemos que
    \begin{align*}
        S(N, V) &= V k_B \left[-\ln\left(1 - \frac{N}{V}\right) + \frac{N}{V}\ln\left(\frac{1 - \frac{N}{V}}{\frac{N}{V}}\right)\right]\\
                &= V k_B\left[\xlnx{\rho} - \xlnx{(1 - \rho)}\right],
    \end{align*}
    portanto a entropia por célula é
    \begin{equation*}
        s_V(\rho) = k_B \left[\xlnx{\rho} - \xlnx{(1 - \rho)}\right].
    \end{equation*}
    O potencial químico absoluto \(\mu\) é dado por
    \begin{equation*}
        \frac{\mu}{T} = - \diffp{S}{N}[V] = - \diffp{s_V}{\rho}[V] = -k_B\left[2 + \ln(\rho - \rho^2)\right].
    \end{equation*}
    Nos limites em que \(\rho \to 0\) e \(\rho \to 1\), temos \(\mu \to \infty\).
\end{proof}
