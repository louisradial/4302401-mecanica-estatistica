\begin{exercício}{Ensemble canônico}{exercício3}
    Mostre que
    \begin{equation*}
        S = k_B \beta^2 \diffp{F}{\beta}
        \quad\text{e}\quad
        c_v = - \beta \diffp{s}{\beta}[v]
    \end{equation*}
    e expresse essas quantidades em termos de \(Z\) e suas derivadas com relação à \(\beta\).
\end{exercício}
\begin{proof}[Resolução]
    Consideremos a distribuição canônica
    \begin{equation*}
        P_i = \frac{1}{Z}e^{-\beta E_i},
    \end{equation*}
    então o valor esperado da energia é dado por
    \begin{equation*}
        \mean{E} = \sum_{i} E_i P_i = -\frac{1}{Z} \diffp*{\sum_{i}{e^{-\beta E_i}}}{\beta} = - \diffp{\ln Z}{\beta},
    \end{equation*}
    e o seu segundo momento por
    \begin{equation*}
        \mean{E^2} = \sum_{i} E_i^2 P_i = \frac{1}{Z}\diffp[2]{\sum_i{e^{-\beta E_i}}}{\beta} = \frac{1}{Z} \diffp[2]{Z}{\beta}.
    \end{equation*}
    Assim, a variância da energia é
    \begin{equation*}
        \mean{E^2} - \mean{E}^2 = \frac{1}{Z}\diffp[2]{Z}{\beta} - \frac{1}{Z^2} \left(\diffp{Z}{\beta}\right)^2 = \diffp*{\left(\frac{1}{Z}\diffp{Z}{\beta}\right)}{\beta} = - \diffp{\mean{E}}{\beta},
    \end{equation*}
    portanto identificando o valor esperado da energia com a energia interna, temos
    \begin{equation*}
        \mean{E^2} - \mean{E}^2 = - \diffp{U}{\beta} = -\diffp{U}{T} \diff{T}{\beta} = \frac{1}{k_B \beta^2} N c_V,
    \end{equation*}
    onde \(c_V = \diffp{u}{T}\) é o calor específico. Então, podemos escrever
    \begin{equation*}
        c_V = \frac{k_B \beta^2}{N} \diffp[2]{\ln Z}{\beta}.
    \end{equation*}

    Da relação entre a energia livre de Helmholtz \(F\) e a função de partição, temos
    \begin{equation*}
        \mean{E} = \diffp{(\beta F)}{\beta} = F + \frac{1}{k_B T} \diffp{F}{T} \diff{T}{\beta} = F - T \diffp{F}{T},
    \end{equation*}
    portanto de \(U = F + TS\), segue que
    \begin{equation*}
        S = - \diffp{F}{T} = -\diffp{F}{\beta}\diff{\beta}{T} = k_B \beta^2 \diffp{F}{\beta}.
    \end{equation*}
    Notemos agora que
    \begin{equation*}
        S = -k_B \beta^2 \diffp*{\left(\beta^{-1} \ln Z\right)}{\beta} = k_B \ln Z - k_B \beta \diffp{\ln{Z}}{\beta},
    \end{equation*}
    portanto
    \begin{equation*}
        \diffp{S}{\beta} = -k_B \beta \diffp[2]{\ln Z}{\beta} = -\frac{k_B \beta N c_V}{k_B \beta^2},
    \end{equation*}
    isto é,
    \begin{equation*}
        c_V = -\beta \diffp{s}{\beta},
    \end{equation*}
    onde \(s = \frac{S}{N}.\)
\end{proof}
