\begin{exercício}{Cristal paramagnético}{exercício8}
    Um cristal paramagnético é composto de \(N\) íons magnéticos. Quando sujeito a um campo \(\vetor{H}\), cada íon contribui para a energia total com energia \(\epsilon_j = -\alpha j\), em que \(\alpha = g \mu_B H\), sendo \(g\) o fator de Landé, \(\mu_B\) o magnéton de Bohr, e \(j\) toma os valores \(-J, -J + 1, \dots, J-1, J\).
    \begin{enumerate}[label=(\alph*)]
        \item Ache a função de partição por partícula \(\zeta(T, H)\).
        \item Ache a energia livre de Helmholtz \(F(T, H, N)\).
        \item Ache a magnetização por íon e faça um gráfico qualitativo de \(m \times H\) e discuta os resultados.
    \end{enumerate}
\end{exercício}
\begin{proof}[Resolução]
    A função de partição por partícula é dada por
    \begin{align*}
        \zeta = \sum_{j = -J}^J e^{\beta \alpha j} &= e^{\beta \alpha J} \sum_{n=0}^{2J} e^{-\beta \alpha n}\\
                                                    &= \frac{e^{-\beta \alpha(2J + 1)} - 1}{e^{-\beta \alpha} - 1}e^{\beta \alpha J}\\
                                                    &= \frac{e^{-\beta \alpha(J+1)} - e^{\beta \alpha J}}{e^{-\beta \alpha} - 1}\\
                                                    &= \frac{e^{\beta \alpha \left(J + \frac12\right)} - e^{-\beta \alpha\left(J + \frac12\right)}}{e^{\frac12 \beta \alpha} - e^{-\frac12 \beta \alpha}}\\
                                                    &= \frac{\sinh\left[\beta \alpha \left(J + \frac12\right)\right]}{\sinh\left(\frac12\beta \alpha\right)}.
    \end{align*}
    Assim, a energia livre de Helmholtz é
    \begin{equation*}
        F = - \frac{N\ln \zeta}{\beta} = \frac{N}{\beta}\ln\left\{\frac{\sinh\left(\frac12 \beta \alpha\right)}{\sinh\left[\beta \alpha (J + \frac12)\right]}\right\} = N k_B T\ln\left\{\frac{\sinh\left(\frac{g \mu_B H}{2k_BT}\right)}{\sinh\left[\frac{(2J+1)g \mu_B H}{2k_BT}\right]}\right\}.
    \end{equation*}
    Portanto, a magnetização por íon é dada por
    \begin{equation*}
        m = -\frac{1}{N}\diffp{F}{H} = \frac12g \mu_B\left\{(2J + 1)\coth\left[\frac{(2J + 1)g \mu_B H}{2 k_B T}\right] - \coth\left(\frac{g \mu_B H}{2 k_B T}\right)\right\}
    \end{equation*}

    \begin{figure}[!ht]
        \centering
        \begin{tikzpicture}
            \begin{axis}[
                width=0.8\linewidth,
                height=0.25\textheight,
                xmin=-5.5, xmax=5.5,
                ymin=-10,ymax=10,
                domain=-6:6,
                samples=500,
                axis lines=middle,
                xlabel={\(H\)},
                % xlabel near ticks,
                % ylabel near ticks,
                ylabel={\(m\)},
                legend pos=south east,
                ytick=\empty,
                xtick=\empty,
            ]
            \addplot[thick, Mauve] {3/tanh(3*x) - 1/tanh(x)};
            \addlegendentry{\(J = 1\)}
            \addplot[thick, Pink] {5/tanh(5*x) - 1/tanh(x)};
            \addlegendentry{\(J = 2\)}
            \addplot[thick, Peach] {7/tanh(7*x) - 1/tanh(x)};
            \addlegendentry{\(J = 3\)}
            \addplot[thick, Red] {9/tanh(9*x) - 1/tanh(x)};
            \addlegendentry{\(J = 4\)}
            % \addplot[thick, Yellow] {11/tanh(11*x) - 1/tanh(x)};
            % \addlegendentry{\(J = 5\)}
            \end{axis}
        \end{tikzpicture}
        \caption{Magnetização por íon}
    \end{figure}
\end{proof}
