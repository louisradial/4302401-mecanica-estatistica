\begin{exercício}{Cristal paramagnético}{exercício8}
    Um cristal paramagnético é composto de \(N\) íons magnéticos. Quando sujeito a um campo \(\vetor{H}\), cada íon contribui para a energia total com energia \(\epsilon_j = -\alpha j\), em que \(\alpha = g \mu_B H\), sendo \(g\) o fator de Landé, \(\mu_B\) o magnéton de Bohr, e \(j\) toma os valores \(-J, -J + 1, \dots, J-1, J\).
    \begin{enumerate}[label=(\alph*)]
        \item Ache a função de partição por partícula \(\zeta(T, H)\).
        \item Ache a energia livre de Helmholtz \(F(T, H, N)\).
        \item Ache a magnetização por íon e faça um gráfico qualitativo de \(m \times H\) e discuta os resultados.
    \end{enumerate}
\end{exercício}
\begin{proof}[Resolução]

\end{proof}
