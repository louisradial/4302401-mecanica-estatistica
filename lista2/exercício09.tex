\begin{exercício}{Sistema de spins localizados}{exercício9}
    Seja \(N\) um número par. Considere um sistema magnético unidimensional de \(N\) spins localizados, a temperatura \(T\) definido pela Hamiltoniana
    \begin{equation*}
        \mathcal{H} = -J \sum_{i = 1}^{\frac{N}{2}} \sigma_{2i-1}\sigma_{2i} - H \sum_{j=1}^{N} \sigma_j,
    \end{equation*}
    onde os parâmetros \(J\) e \(H\) são positivos e \(\sigma_j = \pm 1\) para qualquer sítio \(j\).
    \begin{enumerate}[label=(\alph*)]
        \item Obtenha a função de partição canônica e determine a energia interna \(u = u(T, H)\). Esboce um gráfico de \(u(T, H = 0)\) contra a temperatura \(T\). Obtenha uma expressão para a entropia por spin \(s = s(T, H)\) e esboce um gráfico de \(s(T, H = 0)\) contra \(T\).
        \item Obtenha expressões para a magnetização por partícula \(m = m(T, H)\) e para a suscetibilidade magnética \(\chi = \chi(T, H) = \diffp{m}{H}[T]\). Esboce um gráfico de \(\chi(T, H = 0)\) contra a temperatura.
    \end{enumerate}
\end{exercício}
\begin{proof}[Resolução]
    Notemos que podemos escrever a Hamiltoniana por
    \begin{equation*}
        \mathcal{H} = -J \sum_{i=1}^{\frac{N}{2}} \sigma_{2i - 1} \sigma_{2i} - H\sum_{j=1}^{\frac{N}{2}} \left(\sigma_{2j-1} + \sigma_{2j}\right) = -\sum_{i = 1}^{\frac{N}{2}} \left[J \sigma_{2i-1} \sigma_{2i} + H\left(\sigma_{2i-1} + \sigma_{2i}\right)\right],
    \end{equation*}
    portanto definindo \(S_i = J \sigma_{2i -1} \sigma_{2i} + H(\sigma_{2i - 1} + \sigma_{2i})\), temos \(\mathcal{H} = - \sum_{i = 1}^{\frac{N}{2}}S_i,\) com \(S_i\) e \(S_j\) independentes desde que \(i \neq j\). Como os possíveis valores de \(S_i\) são \(J + 2H, -J\) e \(J - 2H\), com \(-J\) ocorrendo para \(\sigma_{2i-1} = - \sigma_{2i}\), a função de partição do sistema é
    \begin{equation*}
        Z = \left[e^{\beta (J + 2H)} + 2 e^{-\beta J} + e^{\beta (J - 2H)}\right]^{\frac{N}{2}} = \left[2e^{\beta J}\cosh(2\beta H) + 2 e^{-\beta J}\right]^{\frac{N}{2}}.
    \end{equation*}
    Assim, a energia livre de Helmholtz é dada por
    \begin{equation*}
        F = - \frac{\ln Z}{\beta} = - \frac{N}{2 \beta} \ln\left[2e^{\beta J}\cosh(2\beta H) + 2 e^{-\beta J}\right]
    \end{equation*}
    e
    \begin{equation*}
        u = -\frac{1}{N} \diffp{\ln Z}{\beta} = \frac{1}{N} \diffp{(\beta F)}{\beta} =-\frac{J e^{\beta J} \cosh(2 \beta H) + 2H e^{\beta J}\sinh(2\beta H)-J e^{-\beta J}}{2e^{\beta J}\cosh(2\beta H) +  2e^{-\beta J}}
    \end{equation*}
    é a energia interna por partícula.
    \begin{figure}[!ht]
        \centering
        \begin{tikzpicture}
            \begin{axis}[
                width=0.8\linewidth,
                height=0.25\textheight,
                xmin=0, xmax=5.1,
                ymin=-0.6,ymax=0.2,
                domain=0:6,
                samples=500,
                axis lines=middle,
                xlabel={\(T\)},
                % xlabel near ticks,
                % ylabel near ticks,
                ylabel={\(u\)},
                legend pos=south east,
                ytick={-1/2},
                xtick={0,1,2,3,4,5},
                xticklabels={0, \(\frac{J}{k_B}\), \(\frac{2J}{k_B}\), \(\frac{3J}{k_B}\), \(\frac{4J}{k_B}\), \(\frac{5J}{k_B}\)},
                yticklabels={\(-\frac{J}{2}\)},
                unbounded coords=jump
            ]
            \addplot[thick, Mauve] {-1/2 * tanh(1/x)};
            \end{axis}
        \end{tikzpicture}
        \caption{Energia interna por partícula para \(H = 0\)}
    \end{figure}

    A entropia por spin é dada por
    \begin{equation*}
        s = \frac{u - f}{T} = -\frac{1}{2T}\frac{J e^{\beta J} \cosh(2 \beta H) + 2H e^{\beta J}\sinh(2\beta H)-J e^{-\beta J}}{e^{\beta J}\cosh(2\beta H) +  e^{-\beta J}} + \frac{k_B}{2} \ln\left[2e^{\beta J}\cosh(2\beta H) + 2 e^{-\beta J}\right].
    \end{equation*}

    \begin{figure}[!ht]
        \centering
        \begin{tikzpicture}
            \begin{axis}[
                width=0.8\linewidth,
                height=0.2\textheight,
                xmin=0, xmax=5.1,
                ymin=-0.4,ymax=0.1,
                domain=0:6,
                samples=500,
                axis lines=middle,
                xlabel={\(T\)},
                % xlabel near ticks,
                % ylabel near ticks,
                ylabel={\(s\)},
                legend pos=south east,
                ytick={-ln(2)/2},
                xtick={0,1,2,3,4,5},
                xticklabels={0, \(\frac{J}{k_B}\), \(\frac{2J}{k_B}\), \(\frac{3J}{k_B}\), \(\frac{4J}{k_B}\), \(\frac{5J}{k_B}\)},
                yticklabels={\(-\frac{\ln2}{2}k_B\)},
                unbounded coords=jump
            ]
            \addplot[thick, Mauve] {-1/(2*x) * tanh(1/x) + 1/2 * ln(cosh(1/x))};
            \end{axis}
        \end{tikzpicture}
        \caption{Entropia por spin para \(H = 0\)}
    \end{figure}

    A magnetização por partícula é
    \begin{equation*}
        m = -\frac{1}{N}\diffp{F}{H} = \frac{e^{\beta J}\sinh(2 \beta H)}{e^{\beta J}\cosh(2 \beta H) + e^{-\beta J}} = \frac{\sinh(2 \beta H)}{\cosh(2 \beta H) + e^{-2 \beta J}},
    \end{equation*}
    portanto a suscetibilidade magnética é
    \begin{equation*}
        \chi = \diffp{m}{H} = 2 \beta\frac{e^{-2 \beta J}\cosh(2 \beta H) + 1}{\left[\cosh(2 \beta H) + e^{-2 \beta J}\right]^2}
    \end{equation*}
    \begin{figure}[!ht]
        \centering
        \begin{tikzpicture}
            \begin{axis}[
                width=0.8\linewidth,
                height=0.2\textheight,
                xmin=0, xmax=5.1,
                ymin=0.0,ymax=2.2,
                domain=0.2:6,
                samples=500,
                axis lines=middle,
                xlabel={\(T\)},
                % xlabel near ticks,
                % ylabel near ticks,
                ylabel={\(\chi\)},
                legend pos=south east,
                ytick={-ln(2)/2},
                xtick={0,1,2,3,4,5},
                xticklabels={0, \(\frac{J}{k_B}\), \(\frac{2J}{k_B}\), \(\frac{3J}{k_B}\), \(\frac{4J}{k_B}\), \(\frac{5J}{k_B}\)},
                yticklabels={\(-\frac{\ln2}{2}k_B\)},
                unbounded coords=jump
            ]
            \addplot[thick, Mauve] {2/(x*(exp(-2/x) + 1))};
            \end{axis}
        \end{tikzpicture}
        \caption{Suscetibilidade magnética para \(H = 0\)}
    \end{figure}
\end{proof}
