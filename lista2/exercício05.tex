\begin{exercício}{Sistema de íons magnéticos localizados}{exercício5}
    A energia de um sistema de \(N\) íons magnéticos localizados a temperatura \(T\) na presença de um campo \(\vetor{H}\) pode ser escrita na forma \(\mathcal{H} = D \sum_{i = 1}^N S_i^2 - H \sum_{i = 1}^N S_i\), onde os parâmetros \(D\) e \(H\) são positivos e a variável \(S_i\) pode assumir os valores \(-1\), \(0,\) ou \(+1\) para qualquer sítio \(i\).
    \begin{enumerate}[label=(\alph*)]
        \item Obtenha a função de partição \(Z(T, H, D)\) do sistema.
        \item Obtenha expressões para a energia interna, magnetização e entropia por sítio. Para o campo \(H\) nulo, esboce gráficos da energia interna, magnetização e entropia contra a temperatura, indicando o comportamento destas grandezas nos limites \(T \to 0\) e \(T \to \infty\).

        \item Calcule o valor esperado do \emph{momento de quadrupolo} \(q = \frac{1}{N} \sum_{i=1}^N{S_i^2}\) em função do campo e da temperatura.
    \end{enumerate}
\end{exercício}
\begin{proof}[Resolução]
    Notemos que os íons são não interagentes neste modelo, então
    \begin{equation*}
        Z = \left[e^{\beta(-D + H)} + 1 + e^{\beta(-D - H)}\right]^N = \left[1 + 2e^{-\beta D}\cosh(\beta H)\right]^N
    \end{equation*}
    é a função de partição do sistema. Assim, a energia livre de Helmholtz por sítio é
    \begin{equation*}
        f = -\frac{\ln Z}{N \beta } = -\beta^{-1} \ln\left[1 + 2e^{-\beta D}\cosh(\beta H)\right].
    \end{equation*}

    A energia interna por sítio é dada por
    \begin{equation*}
        u = -\frac{1}{N}\diffp{\ln Z}{\beta} = -\frac{2\left[-D\cosh(\beta H) + H \sinh(\beta H)\right]e^{-\beta D}}{1 + 2e^{-\beta D}\cosh(\beta H)} = \frac{2\left[D \cosh(\beta H) - H\sinh(\beta H)\right]}{e^{\beta D} + 2 \cosh(\beta H)},
    \end{equation*}
    portanto para \(H =0\), temos
    \begin{equation*}
        u = \frac{2 D}{e^{\beta D} + 2},
    \end{equation*}
    com \(u \to 0\) conforme \(T \to 0\) e \(u \to \frac{2D}{3}\) conforme \(T \to \infty\).
    \begin{figure}[!ht]
        \centering
        \begin{tikzpicture}
            \begin{axis}[
                width=0.8\linewidth,
                height=0.25\textheight,
                xmin=0, xmax=10.15,
                ymin=0,ymax=0.75,
                domain=0:10.5,
                samples=500,
                axis lines=middle,
                xlabel={T},
                % xlabel near ticks,
                ylabel={\(u\)},
                legend pos=south east,
                ytick={0, 1/6, 1/3, 1/2, 2/3,5/6},
                xtick={0, 2, 4, 6, 8,10},
                xticklabels={0, \(\frac{2D}{k_B}\), \(\frac{4D}{k_B}\), \(\frac{6D}{k_B}\), \(\frac{8D}{k_B}\), \(\frac{10D}{k_B}\)},
                yticklabels={0,\(\frac{D}{6}\),\( \frac{D}{3}\),\( \frac{D}{2}\),\( \frac{2D}{3}\),\( \frac{5D}{6}\)},
            ]
            \addplot[thick, Mauve] {2/(exp(1/x) + 2)};
            \addplot[thick, dashed] {2/3};
            \end{axis}
        \end{tikzpicture}
        \caption{Energia interna por sítio com campo \(H\) nulo.}
    \end{figure}

    A magnetização por sítio é dada por
    \begin{align*}
        m &= \mean*{\frac{1}{N}\sum_{i=1}^N S_i} = \frac{1}{NZ}\sum_{S_i} \left(\sum_{i=1}^N S_i\right) \exp\left[-\beta \left(D\sum_{i=1}^N S_i^2 - H \sum_{i = 1}^N S_i\right)\right]\\
          &= \frac{1}{NZ \beta} \diffp*{\left\{\sum_{S_i}\exp\left[-\beta\left(D\sum_{i=1}^N S_i^2 - H \sum_{i=1}^N S_i\right)\right] \right\}}{H} = \frac{1}{N \beta} \diffp{\ln Z}{H}\\
          &= -\diffp{f}{H}[T,D] = \frac{2e^{-\beta D}\sinh(\beta H)}{1 + 2e^{-\beta D}\cosh(\beta H)} = \frac{2\sinh(\beta H)}{e^{\beta D} + 2\cosh(\beta H)},
    \end{align*}
    portanto, para \(H = 0\), temos \(m = 0\) para todo \(T \geq 0\).

    A entropia por sítio é dada por
    \begin{equation*}
        s = - \diffp{f}{T} = -\diffp{f}{\beta} \diff{\beta}{T} = k_B \ln\left[1 + 2 e^{-\beta D}\cosh(\beta H)\right] -  \frac{2k_B \beta\left[-D \cosh(\beta H) + 2H\sinh(\beta H)\right]}{e^{\beta D} + 2\cosh(\beta H)},
    \end{equation*}
    portanto para \(H = 0\), temos
    \begin{equation*}
        s = k_B \left[ \ln(1 + 2e^{-\beta D}) + \frac{2 \beta D}{e^{\beta D} + 2}\right],
    \end{equation*}
    portanto \(s \to 0\) conforme \(T \to 0\) e \(s \to k_B \ln(3)\) conforme \(T \to \infty\).
    \begin{figure}[!ht]
        \centering
        \begin{tikzpicture}
            \begin{axis}[
                width=0.8\linewidth,
                height=0.25\textheight,
                xmin=0, xmax=5.15,
                ymin=0,ymax=1.3,
                domain=0:10.5,
                samples=500,
                axis lines=middle,
                xlabel={T},
                % xlabel near ticks,
                ylabel={\(s\)},
                legend pos=south east,
                ytick={0, {ln(3)}},
                xtick={0, 1, 2, 3, 4,5},
                xticklabels={0, \(\frac{D}{k_B}\), \(\frac{2D}{k_B}\), \(\frac{3D}{k_B}\), \(\frac{4D}{k_B}\), \(\frac{5D}{k_B}\)},
                yticklabels={0, \(k_B \ln3\)},
            ]
            \addplot[thick, Mauve] {ln(1 + 2*exp(-1/x)) + (2/x)/(exp(1/x) + 2)};
            \addplot[thick, dashed] {ln(3)};
            \end{axis}
        \end{tikzpicture}
        \caption{Entropia por sítio com campo \(H\) nulo.}
    \end{figure}

    O valor esperado do momento de quadrupolo é dado por
    \begin{align*}
        \mean{q} &= \frac{1}{NZ} \sum_{S_i} \left(\sum_{i=1}^N S_i^2\right) \exp\left[-\beta\left(D \sum_{i=1}^N S_i^2 - H\sum_{i=1}^N S_i\right)\right]\\
                 &= -\frac{1}{N \beta Z} \diffp{Z}{D}[H,T] = \diffp{f}{D}[H,T]\\
                 &= \frac{2 \cosh(\beta H)}{e^{\beta D} + 2\cosh(\beta H)}.
    \end{align*}
\end{proof}
