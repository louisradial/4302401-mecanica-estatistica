\begin{exercício}{Sistema de íons magnéticos localizados}{exercício5}
    A energia de um sistema de \(N\) íons magnéticos localizados a temperatura \(T\) na presença de um campo \(\vetor{H}\) pode ser escrita na forma \(\mathcal{H} = D \sum_{i = 1}^N S_i^2 - H \sum_{i = 1}^N S_i\), onde os parâmetros \(D\) e \(H\) são positivos e a variável \(S_i\) pode assumir os valores \(-1\), \(0,\) ou \(+1\) para qualquer sítio \(i\).
    \begin{enumerate}[label=(\alph*)]
        \item Obtenha a função de partição \(Z(T, H, D)\) do sistema.
        \item Obtenha expressões para a energia interna, magnetização e entropia por sítio. Para o campo \(H\) nulo, esboce gráficos da energia interna, magnetização e entropia contra a temperatura, indicando o comportamento destas grandezas nos limites \(T \to 0\) e \(T \to \infty\).

        \item Calcule o valor esperado do \emph{momento de quadrupolo} \(q = \frac{1}{N} \mean{S_i^2}\) em função do campo e da temperatura.
    \end{enumerate}
\end{exercício}
\begin{proof}[Resolução]

\end{proof}
