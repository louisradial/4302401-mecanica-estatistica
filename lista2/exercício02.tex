\begin{exercício}{Gás de rede}{exercício2}
    Considere um gás de rede constituído por \(N\) partículas distribuídas em \(V\) células, com \(V > N\). Suponha que cada célula possa estar vazia ou ocupada por uma única partícula.
    \begin{enumerate}[label=(\alph*)]
        \item Ache a expressão para a densidade de estados \(\Omega(N, V)\) e a entropia por partícula \(s(v)\), onde \(v = \frac{V}{N}\).
        \item Obtenha a pressão \(p\) e escreva uma expressão em termos da densidade \(\rho = \frac1v\). Mostre que o primeiro termo corresponde à lei de Boyle dos gases ideais.
        \item Obtenha uma expressão para o potencial químico absoluto \(\mu\) em função da densidade. Qual é o comportamento do potencial químico nos limites \(\rho \to 0\) e \(\rho \to 1\)?
    \end{enumerate}
\end{exercício}
\begin{proof}[Resolução]

\end{proof}
