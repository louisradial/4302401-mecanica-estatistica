\begin{exercício}{Superfície adsorvente}{exercício9}
    A uma temperatura \(T\), uma superfície com \(N_0\) centros de adsorção tem \(N \leq N_0\) moléculas adsorvidas. Supondo que não haja interação entre as moléculas, mostre que o potencial químico do gás adsorvido pode ser escrito da seguinte forma
    \begin{equation*}
        \mu = k_BT \ln\left[\frac{N}{(N_0 - N) a(T)}\right].
    \end{equation*}
    Qual seria a interpretação da função \(a(T)\)?
\end{exercício}
\begin{proof}[Resolução]
    Assumindo que cada molécula adsorvida contribui com uma energia \(\epsilon\), a função de partição canônica é
    \begin{equation*}
        Z(V, T, N) = \sum_{j} e^{-N \beta \epsilon} = \binom{N_0}{N} e^{-N \beta \epsilon}.
    \end{equation*}
    Portanto, a grande função de partição é
    \begin{equation*}
        \Xi(T, V, z) = \sum_{N=0}^{N_0} z^N Z(V, T, N) = \sum_{N = 0}^{N_0} \binom{N_0}{N} \left(z e^{-\beta \epsilon}\right)^{N} = \left(1 + ze^{-\beta \epsilon}\right)^{N_0}.
    \end{equation*}
    O número médio de partículas é dado por
    \begin{equation*}
        \mean{N} = \Xi^{-1}\sum_{N=0}^{N_0} Nz^N Z = z\diffp{\ln \Xi}{z} = \frac{e^{-\beta \epsilon}z N_0}{1 + z e^{-\beta \epsilon}} = \frac{z N_0}{e^{\beta \epsilon} + z}
    \end{equation*}
    donde segue que
    \begin{equation*}
        e^{\beta \mu} = z = \frac{e^{\beta \epsilon}\mean{N}}{N_0 - \mean{N}} \implies \mu = k_BT \ln\left[\frac{\mean{N}}{\left(N_0 - \mean{N}\right)\exp\left(-\frac{\epsilon}{k_B T}\right)}\right]
    \end{equation*}
    é o potencial químico. Notemos que \(\exp\left(-\frac{\epsilon}{k_BT}\right)\) é a função de partição canônica para uma única molécula adsorvida dividida pelo número de centros de adsorção.
\end{proof}
