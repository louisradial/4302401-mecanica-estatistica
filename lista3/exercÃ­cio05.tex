\begin{exercício}{Gás de moléculas diatômicas fracamente interagente}{exercício5}
    Considere um sistema clássico de \(N\) moléculas diatômicas muito fracamente interagentes, dentro de um recipiente de volume \(V\) a uma dada temperatura \(T\). O hamiltoniano de uma única molécula é dada por
    \begin{equation*}
        \mathcal{H} = \frac{p_1^2 + p_2^2}{2m} + \frac{K}{2}\norm{\vetor{r}_1 - \vetor{r}_2}^2,
    \end{equation*}
    onde \(K > 0\) é uma constante elástica. Ache as propriedades termodinâmicas e obtenha uma expressão para a energia livre de Helmholtz deste sistema. Calcule o calor específico a volume constante. Calcule o diâmetro molecular médio \(D = \sqrt{\mean{\norm{\vetor{r}_1 - \vetor{r}_2}^2}}\).
\end{exercício}
\begin{proof}[Resolução]

\end{proof}
