\begin{exercício}{Gás clássico em contato com reservatório térmico e de partículas}{exercício7}
    Considere o gás clássico ultra-relativístico cuja Hamiltoniana por partícula é dada por
    \begin{equation*}
        \mathcal{H} = c\sqrt{p_x^2 + p_y^2 + p_z^2}
    \end{equation*}
    em contato agora com um reservatório térmico e de partículas. Obtenha a grande função de partição e o potencial grande canônico desse sistema.
\end{exercício}
\begin{proof}[Resolução]
    Pelo \cref{ex:exercício3}, a grande função de partição é dada por
    \begin{equation*}
        \Xi(V, T, \mu) = \sum_{N = 0}^\infty e^{\beta \mu N} Z(V, T, N) = \sum_{N=0}^\infty \frac{1}{N!}\left[\frac{8\pi Ve^{\beta \mu}(k_BT)^3}{h^3 c^3}\right]^N = \exp\left[\frac{8\pi Ve^{\beta \mu}(k_BT)^3}{h^3 c^3}\right].
    \end{equation*}
    Dessa forma,
    \begin{equation*}
        \Phi(V, T, \mu) = -k_B T \ln\Xi(V, T, \mu) = -\frac{8\pi V e^{\beta \mu}(k_BT)^4}{h^3 c^3}
    \end{equation*}
    é o potencial grande canônico do sistema.
\end{proof}
