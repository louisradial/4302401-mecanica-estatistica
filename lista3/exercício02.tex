\begin{exercício}{Gás de partículas relativísticas não interagentes}{exercício2}
    Um sistema de \(N\) partículas clássicas ultra-relativísticas, dentro de um recipiente de volume \(V\), a uma temperatura \(T\), onde cada partícula é definida pela Hamiltoniana
    \begin{equation*}
        \mathcal{H} = c\sqrt{p_x^2 + p_y^2 + p_z^2}.
    \end{equation*}
    Obtenha uma expressão para a função canônica de partição. Calcule a entropia como função da temperatura e do volume específico. Qual a expressão do calor específico a volume constante?
\end{exercício}
\begin{proof}[Resolução]
    A função de partição por partícula é dada por
    \begin{align*}
        \zeta &= \frac{1}{h^3}\iiint_{V} \dln3x \iiint_{\mathbb{R}^3} \dln3p \exp\left(-\beta c \norm{\vetor{p}}\right)\\
              &= \frac{V}{h^3} \int_0^\infty \dli{p} \int_0^{\pi} p \dli{\theta} \int_0^{2\pi} p\sin\theta \dli{\phi} \exp(- \beta pc)\\
              &= \frac{4\pi V}{h^3} \int_0^\infty \dli{p} \exp(- \beta pc)\\
              &= \frac{4\pi V}{ \beta h^3c},
    \end{align*}
    portanto a função de partição canônica é
    \begin{equation*}
        Z = \frac{\zeta^N}{N!} = \frac{1}{N!}\left(\frac{4\pi V}{ \beta c h^3}\right)^N.
    \end{equation*}
    Assim, a energia livre de Helmholtz é dada por
    \begin{align*}
        F = - k_BT \ln Z &= -k_BT\left(N \ln{\zeta} - \ln{N!}\right)\\
                         &= -N k_BT\left[1 + \ln\left(\frac{\zeta}{N}\right)\right]\\
                         &= -N k_B T\left[1 + \ln\left(\frac{4\pi V k_B T}{N h^3c}\right)\right]
    \end{align*}
    e a energia interna por
    \begin{equation*}
        U = \diffp{\beta F}{\beta} = -N \diffp*{\ln\left(\frac{4\pi V}{N \beta h^3 c}\right)}{\beta} = N k_B T.
    \end{equation*}
    Dessa forma, a entropia é
    \begin{equation*}
        S = \frac{U - F}{T} = N k_B\left[2 + \ln\left(\frac{4\pi V k_BT}{N h^3 c}\right)\right]
    \end{equation*}
    e
    \begin{equation*}
        c_V = \lim_{N \to \infty} \frac{1}{N}\diffp{U}{T}[V,N] = k_B
    \end{equation*}
    é o calor específico a volume constante.
\end{proof}
