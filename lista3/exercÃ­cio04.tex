\todo[Trocar rot, tr, m por mathrm]
\begin{exercício}{Gás de moléculas diatômicas rígidas fracamente interagentes}{exercício4}
    Desprezando o movimento vibracional, uma molécula diatômica pode ser tratada como um rotor rígido tridimensional. O hamiltoniano \(\mathcal{H}_m\) da molécula é escrito na forma de um termo translacional \(\mathcal{H}_{tr}\) somado de um termo rotacional \(\mathcal{H}_{rot}\), isto é, \(\mathcal{H}_m=\mathcal{H}_{tr} + \mathcal{H}_{rot}\). Considere um sistema de \(N\) moléculas dessa natureza, muito fracamente interagentes, dentro de um recipiente de volume \(V\), a uma dada temperatura \(T\).
    \begin{enumerate}[label=(\alph*)]
        \item Obtenha a expressão para \(\mathcal{H}_{rot}\) em coordenadas esféricas. Obtenha a contribuição do termo rotacional para a função de partição e uma expressão para o calor específico a volume constante.
        \item Suponha agora que cada molécula possua momento de dipolo permanente \(\vetor{d}\) dirigido ao longo do eixo do rotor e que o sistema se encontre encontre na presença de um campo elétrico externo \(\vetor{E}\), de forma que na Hamiltoniana é acrescido um termo dado por \(-d E \cos\theta\). Obtenha as propriedades termodinâmicas do gás. Ache a polarização por molécula \(P = N\mean{d \cos\theta}\) e a suscetibilidade elétrica \(\chi = \diffp{P}{E}\).
    \end{enumerate}
\end{exercício}
\begin{proof}[Resolução]

\end{proof}
