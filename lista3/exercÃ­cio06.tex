\begin{exercício}{Entropia no ensemble grande canônico}{exercício6}
    Mostre que a entropia no ensemble grande canônico pode ser escrita na forma
    \begin{equation*}
        S = - k_B\sum_j P_j \ln{P_j},
    \end{equation*}
    onde
    \begin{equation*}
        P_j = \Xi^{-1} \exp\left(-\beta E_j + \beta \mu N_j\right)
    \end{equation*}
    é a expressão para a probabilidade \(P_j\).
\end{exercício}
\begin{proof}[Resolução]
    O grande potencial é dado por \(\Phi = -k_B T \ln \Xi(V, T, N)\), então a entropia é
    \begin{align*}
        S = - \diffp{\Phi}{T}[V, N] &= k_B \ln \Xi + k_B T \Xi^{-1} \diffp{\Xi}{\beta}[V,N]\diff{\beta}{T}\\
                                    &= k_B \ln \Xi - k_B \Xi^{-1} \sum_{j} \frac{\mu N_j - E_j}{k_BT} \exp\left(\frac{\mu N_j - E_j}{k_B T}\right)\\
                                    &= k_B \ln \Xi - k_B \sum_j \ln\left(\Xi P_j\right) P_j\\
                                    &= -k_B \sum_j P_j \ln P_j,
    \end{align*}
    como desejado.
\end{proof}
