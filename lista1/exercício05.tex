\begin{exercício}{Função geradora para a distribuição de Poisson}{exercício5}
    A probabilidade de que um evento ocorra \(k\) vezes num total de \(n\) tentativas é dada por
    \begin{equation*}
        P(k) = \frac{\lambda^k}{k!}\exp(-\lambda).
    \end{equation*}
    Obtenha a função geradora \(\phi(t)\) para a distribuição de Poisson bem como a média \(\mean{k}\) e a variância \(\sigma^2 = \mean{k^2} - \mean{k}^2\).
\end{exercício}
\begin{proof}[Resolução]
    A função geradora é dada por \(\phi(t) = \mean{e^{tk}}\), isto é,
    \begin{align*}
        \phi(t) &= e^{-\lambda}\sum_{k = 0}^\infty \frac{\lambda^k}{k!}e^{tk}\\
                &= e^{-\lambda} \sum_{k = 0}^\infty \frac{(\lambda e^t)^k}{k!}\\
                &= \exp[\lambda (e^t - 1)].
    \end{align*}
    Notemos que \(\ln \phi = \lambda (e^t -1)\), então
    \begin{equation*}
        \diff{\phi}{t} = \lambda e^t \phi(t) \quad\text{e}\quad
        \diff[2]{\phi}{t} = \left(\lambda e^t  + \lambda^2 e^{2t}\right)\phi(t)
    \end{equation*}
    são as derivadas primeira e segunda da função geradora. Logo,
    \begin{equation*}
        \mean{k} = \lambda\quad\text{e}\quad
        \mean{k^2} = \lambda + \lambda^2
    \end{equation*}
    portanto a variância é \(\sigma^2 = \lambda.\)
\end{proof}
