\begin{exercício}{Coleção de íons paramagnéticos}{exercício9}
    Considere agora uma coleção de \(N\) íons paramagnéticos na presença de um campo magnético \(H\). Da mesma forma que no \cref{ex:exercício6} cada íon é permitido estar apenas na direção paralela ou antiparalela ao campo magnético. Como ficariam as expressões para a magnetização \(M\) e variância totais? No limite de \(N\) muito grande, como seria a forma da distribuição de probabilidades \(P_N(M)\) da variável aleatória \(M\)?
\end{exercício}
\begin{proof}[Resolução]

\end{proof}
