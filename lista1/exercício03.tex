\begin{exercício}{Comparação entre distribuição binomial e gaussiana}{exercício3}
    Considere uma distribuição binomial com \(N = 60\) e probabilidade de sucesso \(p = \frac23\).
    \begin{enumerate}[label=(\alph*)]
        \item Trace um gráfico de \(P(k)\) contra \(\frac{k}{N}\).
        \item Obtenha a distribuição gaussiana correspondente, \(p_G(k)\), com os mesmos valores de valor esperado e variância da distribuição binomial.
        \item Repita os itens (a) e (b) para \(N = 15\) e \(N = 30\). Há modificações sensíveis?
    \end{enumerate}
\end{exercício}
\begin{proof}[Resolução]
    A função geradora dos momentos \(\phi(t)\) de uma distribuição binomial de \(N\) ensaios com probabilidade \(p\) é dada por
    \begin{align*}
        \phi(t) = \mean{e^{kt}} &= \sum_{k=0}^N e^{tk}\binom{N}{k} p^{k}(1-p)^{N-k}\\
                                &= \sum_{k=0}  \binom{N}{k} (e^t p)^k (1 - p)^{N -k}\\
                                &= (e^t p + 1 - p)^N.
    \end{align*}
    Assim, os momentos da distribuição binomial são dados por
    \begin{equation*}
        \mean{k} = \phi'(0) = Np
        \quad\text{e}\quad
        \mean{k^2} = \phi''(0) = Np(1 - p) + N^2p^2.
    \end{equation*}
    Portanto, a distribuição gaussiana correspondente é
    \begin{equation*}
        p_G(k) = \frac{1}{\sqrt{2\pi Np(1 - p)}} \exp\left[-\frac{(x - Np)^2}{2Np(1-p)}\right],
    \end{equation*}
    pois a variância da distribuição binomial é \(\sigma^2 = \mean{k^2} - \mean{k}^2 = Np(1 - p)\).
    \begin{figure}[!h]
        \centering
        \includegraphics[height=0.3\textheight]{figuras/exercício03.png}
        \caption{Comparação entre distribuições binomiais e suas distribuições gaussianas correspondentes}
    \end{figure}
\end{proof}
