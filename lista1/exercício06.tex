\begin{exercício}{Íon paramagnético}{exercício6}
    Considere um íon paramagnético na presença de um campo magnético \(H\). O íon é permitido estar apenas na direção paralela ou antiparalela ao campo magnético, onde em cada caso (estado microscópico), podemos associar uma energia \(E_i = -H \sigma_i\), onde \(\sigma_i = 1\) (paralelo) ou \(\sigma_i = -1\) (antiparalelo). A distribuição de probabilidade associada a cada estado é dada por \(P(\sigma_i) = \zeta^{-1} \exp(-\beta E_i),\) com \(\beta = \frac{1}{k_BT},\) sendo \(k_B\) a constante de Boltzmann.
    \begin{enumerate}[label=(\alph*)]
        \item Ache a expressão para \(\zeta(\beta, H)\).
        \item Calcule a energia média \(\mean{E_i}\) do íon e faça um gráfico de \(\mean{E_i} \times T\);
        \item Mostre que a magnetização do íon \(m = \mean{\sigma_i}\) pode ser calculada a partir da expressão \(m = k_B T \diffp*{\ln \zeta}{H}\). Encontre uma expressão para \(m(T, H)\) e mostre que \(m(T, 0) = 0\), resultado esperado para um material paramagnético.
        \item Obtenha uma expressão para a variância \(\chi = \mean{\sigma_i^2} - \mean{\sigma_i}^2\).
    \end{enumerate}
\end{exercício}
\begin{proof}[Resolução]
    Para que a distribuição esteja normalizada, devemos ter \(\zeta = 2\cosh(\beta H)\). Assim, a energia média do íon \(\mean{E_i}\) é
    \begin{equation*}
        \mean{E_i} = -\sum_{\sigma_i \in \set{-1, 1}} \frac{H \sigma_i}{\zeta}\exp(\beta H \sigma_i) = -H\tanh(\beta H).
    \end{equation*}

    \begin{figure}[!h]
        \centering
        \begin{tikzpicture}
            \begin{axis}[
                width=0.8\linewidth,
                height=0.25\textheight,
                xmin=0, xmax=5,
                ymin=-1.1,ymax=0.25,
                domain=0:5,
                samples=500,
                axis lines=middle,
                xlabel={$T$},
                ylabel={$\mean{E_i}$},
                legend pos=north east,
                ytick=\empty,
                xtick=\empty
            ]
                \addplot[thick, Mauve] {- tanh(1/\x)};
            \end{axis}
        \end{tikzpicture}
        \caption{Energia média \(\mean{E_i}\) do íon em função da temperatura \(T\).}
    \end{figure}

    Notemos que
    \begin{equation*}
        \mean{\sigma_i} = \sum_{\sigma_i\in\set{-1,1}} \frac{\sigma_i}{\zeta} \exp(\beta H \sigma_i) = -\frac{1}{H}\mean{E_i} = \tanh(\beta H)
    \end{equation*}
    é o primeiro momento de \(\sigma_i\), que chamaremos de magnetização \(m\) e podemos escrever
    \begin{equation*}
        m(T, H) = \tanh\left(\frac{H}{k_B T}\right).
    \end{equation*}

    Por outro lado, temos
    \begin{equation*}
        \diffp*{\ln \zeta}{H} = \zeta^{-1} \diffp{\zeta}{H} = \beta \tanh(\beta H) = \beta m,
    \end{equation*}
    isto é,
    \begin{equation*}
        m = k_B T \diffp*{\ln \zeta}{H}.
    \end{equation*}
    É fácil constatar a partir da expressão que relaciona a magnetização com o valor esperado de \(\sigma_i\) que, independente do valor de \(\beta\), a magnetização se anula na ausência do campo magnético, isto é, \(m(T, 0) = 0\).

    Veja que
    \begin{equation*}
        \mean{\sigma_i^2} = \sum_{\sigma_i\in \set{-1,1}} \frac{\sigma_i^2}{\zeta}\exp(\beta H \sigma_i) = \sum_{\sigma_i \in \set{-1,1}} \frac{1}{\zeta} \exp(\beta H \sigma_i) = 1,
    \end{equation*}
    portanto
    \begin{equation*}
        \chi = \mean{\sigma_i}^2 - \mean{\sigma_i}^2 = 1 - m^2 = \sech^2(\beta H)
    \end{equation*}
    é a variância de \(\sigma_i\).
\end{proof}
