\begin{exercício}{Distribuição de Maxwell para as velocidades moleculares}{exercício7}
    Considere a densidade de probabilidade dada por
    \begin{equation*}
        \rho(v_x,v_y,v_z) = \left(\frac{\beta m}{2\pi}\right)^{\frac32}\exp\left[-\beta m \left(v_x^2 + v_y^2 + v_z^2\right)\right],
    \end{equation*}
    em que \(\vetor{v} = v_x\vetor{e}_x+v_y\vetor{e}_y+v_z\vetor{e}_z\) é a velocidade e \(\beta = \frac{1}{k_BT}\), sendo \(k_B\) a constante de Boltzmann e \(T\) a temperatura absoluta.
    \begin{enumerate}[label=(\alph*)]
        \item Mostre que
            \begin{equation*}
                \rho(v, \theta, \phi) = v^2 \sin\theta\left(\frac{\beta m}{2\pi}\right)^{\frac32}\exp\left(-\beta m v^2\right),
            \end{equation*}
            onde \(v = \norm{\vetor{v}}\), \(v\cos\theta = v_z\) e \(v\sin\theta\cos\phi = v_x\).
        \item Obtenha a densidade de probabilidade marginal \(\rho_v(v)\).
    \end{enumerate}
\end{exercício}
\begin{proof}[Resolução]

\end{proof}

