\begin{exercício}{Distribuição de Poisson}{exercício4}
    A probabilidade de que um evento caracterizado pela probabilidade \(p\) ocorra \(k\) vezes num total de \(n\) tentativas é dada pela distribuição binomial. Considere uma situação em que \(p\) seja pequeno e a distribuição seja apreciavelmente diferente de zero apenas para \(k \ll n\). Nessas circunstâncias, mostre que \(P(k) \sim \frac{\lambda^k}{k!}\exp(-\lambda)\), onde \(\lambda = np\) é o número médio de eventos.
\end{exercício}
\begin{proof}[Resolução]
    Pela distribuição binomial, temos
    \begin{equation*}
        P(k) = \binom{n}{k} p^{k}(1-p)^{n - k} = \frac{p^k}{k!} (1 -p)^{n-k} \prod_{\ell = 0}^{k-1} (n - \ell)
    \end{equation*}
    como a probabilidade do evento ocorrer \(k\) vezes. Em termos de \(\lambda = np\), temos
    \begin{equation*}
        P(k) = \frac{\lambda^k}{k!} \left(1 - \frac{\lambda}{n}\right)^{n - k}\prod_{\ell = 0}^{k - 1} \left(1 - \frac{\ell}{n}\right).
    \end{equation*}
    Sob a hipótese \(k \ll n\), obtemos
    \begin{equation*}
        P(k) \simeq \frac{\lambda^k}{k!} \left(1 - \frac{\lambda}{n}\right)^{n},
    \end{equation*}
    de modo que
    \begin{equation*}
        \lim_{n \to \infty} P(k) = \frac{\lambda^k}{k!} \exp(-\lambda),
    \end{equation*}
    como desejado.
\end{proof}
