\begin{exercício}{Distribuição de Poisson}{exercício4}
    A probabilidade de que um evento caracterizado pela probabilidade \(p\) ocorra \(k\) vezes num total de \(n\) tentativas é dada pela distribuição binomial. Considere uma situação em que \(p\) seja pequeno e a distribuição seja apreciavelmente diferente de zero apenas para \(k \ll n\). Nessas circunstâncias, mostre que \(P(k) \sim \frac{\lambda^k}{k!}\exp(-\lambda)\), onde \(\lambda = np\) é o número médio de eventos.
\end{exercício}
\begin{proof}[Resolução]

\end{proof}
