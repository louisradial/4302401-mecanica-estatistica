\begin{exercício}{Íon paramagnético}{exercício6}
    Considere um íon paramagnético na presença de um campo magnético \(H\). O íon é permitido estar apenas na direção paralela ou antiparalela ao campo magnético, onde em cada caso (estado microscópico), podemos associar uma energia \(E_i = -H \sigma_i\), onde \(\sigma_i = 1\) (paralelo) ou \(\sigma_i = -1\) (antiparalelo). A distribuição de probabilidade associada a cada estado é dada por \(P(\sigma_i) = \zeta^{-1} \exp(-\beta E_i),\) com \(\beta = \frac{1}{k_BT},\) sendo \(k_B\) a constante de Boltzmann.
    \begin{enumerate}[label=(\alph*)]
        \item Ache a expressão para \(\zeta(\beta, H)\).
        \item Calcule a energia média \(\mean{E_i}\) do íon e faça um gráfico de \(\mean{E_i} \times T\);
        \item Mostre que a magnetização do íon \(m = \mean{\sigma_i}\) pode ser calculada a partir da expressão \(m = k_B T \diffp*{\ln \zeta}{H}\). Encontre uma expressão para \(m(T, H)\) e mostre que \(m(T, 0) = 0\), resultado esperado para um material paramagnético.
        \item Obtenha uma expressão para a variância \(\chi = \mean{\sigma_i^2} - \mean{\sigma_i}^2\).
    \end{enumerate}
\end{exercício}
\begin{proof}[Resolução]

\end{proof}
