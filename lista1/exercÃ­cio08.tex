\begin{exercício}{Caminho aleatório unidimensional}{exercício8}
    Num caminho aleatório em uma dimensão, depois de \(N\) passos a partir da origem, a posição é dada por \(x = \sum_{j=1}^N s_j\), onde \(\set{s_j}\) é o conjunto de variáveis aleatórias independentes, identicamente distribuídas de acordo com a distribuição de probabilidades
    \begin{equation*}
        P(s) = \frac{1}{\sqrt{2\pi \sigma^2}} \exp{\left(-\frac{(s - \ell)^2}{2 \sigma^2}\right)},
    \end{equation*}
    onde \(\sigma\) e \(\ell\) são constantes positivas.
    \begin{enumerate}[label=(\alph*)]
        \item Ache o deslocamento médio a partir da origem depois de \(N\) passos.
        \item Qual é o valor do desvio quadrático médio da variável aleatória \(x\)?
    \end{enumerate}
\end{exercício}
\begin{proof}[Resolução]

\end{proof}


